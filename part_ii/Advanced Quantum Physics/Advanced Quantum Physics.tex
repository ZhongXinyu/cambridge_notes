\documentclass[12pt,a4paper]{article}

\usepackage{import}
\import{../Template/}{format.tex}

\newcommand{\topic}{Advanced Quantum Physics}

\begin{document}

\begin{titlepage}
    \maketitle
\end{titlepage}

\tableofcontents

\newpage

\begin{abstract}
\noindent
Abstract of this course
\end{abstract}

\section{Revision}
\section{Perturbation Theory}
\subsection{Time-Independent Perturbation Theory}
\subsection{First-order Perturbation Theory}
$$
\begin{aligned} & E_n \simeq E_n^{(0)}+\left\langle n^{(0)}\left|\hat{H}^{(1)}\right| n^{(0)}\right\rangle \\ & |n\rangle \simeq\left|n^{(0)}\right\rangle+\sum_{m \neq n}\left|m^{(0)}\right\rangle \frac{\left\langle m^{(0)}\left|\hat{H}^{(1)}\right| n^{(0)}\right\rangle}{E_n^{(0)}-E_m^{(0)}}\end{aligned}
$$
\subsection{Second-order Perturbation Theory}
$$E_n^{(2)}=\sum_{m \neq n} \frac{\left|\left\langle m^{(0)}\left|\hat{H}^{(1)}\right| n^{(0)}\right\rangle\right|^2}{E_n^{(0)}-E_m^{(0)}}$$

\begin{example}
    {Infinite square well with central bump}
    {...}
\end{example}
\begin{example}
    {Infinite square well in an electric field}
    {...}
\end{example}
\begin{example}
    {Harmonic Oscillator + Linear perturbation}
    {...}
\end{example}
\begin{example}
    {Van der Waals Interaction}
    {...}
\end{example}
\subsection{Degenerate Perturbation Theory}
\begin{example}
    {Perturbed 2D infinite square well}
    {...}
\end{example}
\subsection{Variation Method}
\begin{example}
    {Hydrogen atom ground state energy}
    {...}
\end{example}
\subsubsection{Rayleigh-Ritz Method}
\begin{example}
    {Hydrogen atom with finite proton mass}
    {...}
\end{example}
\section{Wigner-Eckart theorem}
Let $\hat{K}$ be a scalar operator with respect to an angular momentum operator $\hat{\boldsymbol{J}}$ :
$$
[\hat{\boldsymbol{J}}, \hat{K}]=0
$$
Then the Wigner-Eckart theorem states that matrix elements of $\hat{K}$ taken between total angular momentum eigenstates $\left|\alpha^{\prime \prime} j^{\prime \prime} m^{\prime \prime}\right\rangle$ and $\left|\alpha^{\prime} j^{\prime} m^{\prime}\right\rangle$ must be of the form
$$
\left\langle\alpha^{\prime \prime} j^{\prime \prime} m^{\prime \prime}|\hat{K}| \alpha^{\prime} j^{\prime} m^{\prime}\right\rangle=C\left(\alpha^{\prime \prime} \alpha^{\prime} ; j^{\prime}\right) \delta_{j^{\prime \prime} j^{\prime}} \delta_{m^{\prime \prime} m^{\prime}}
$$
where $C\left(\alpha^{\prime \prime} \alpha^{\prime} ; j^{\prime}\right)$ is a complex constant known as the reduced matrix element which is independent of the quantum numbers $m^{\prime \prime}$ and $m^{\prime}$. The quantities $\alpha^{\prime \prime}$ and $\alpha^{\prime}$ collectively label all other quantum numbers needed to uniquely identify the angular momentum eigenstates involved.
\subsection{Consequence of Wigner-Eckart theorem (Scalar Hamiltonian)}
Since $\hat{H}$ and $\hat{\boldsymbol{J}}$ commute, they possess a simultaneous set of eigenstates $|\alpha j m\rangle$, where " $\alpha$ " represents all other quantum numbers needed to uniquely identify a particular energy eigenstate of $\hat{H}$. From the Wigner-Eckart theorem, the matrix elements of $\hat{H}$ between these eigenstates must be of the form
$$
\left\langle\alpha^{\prime \prime} j^{\prime \prime} m^{\prime \prime}|\hat{H}| \alpha^{\prime} j^{\prime} m^{\prime}\right\rangle=\left\langle\alpha^{\prime \prime} j^{\prime \prime}\|\hat{H}\| \alpha^{\prime} j^{\prime}\right\rangle \delta_{j^{\prime \prime} j^{\prime}} \delta_{m^{\prime \prime} m^{\prime}}
$$
where the reduced matrix element $\left\langle\alpha^{\prime \prime} j^{\prime \prime}\|\hat{H}\| \alpha^{\prime} j^{\prime}\right\rangle$ is a constant, independent of the quantum number $m$. In particular, the expectation values of $\hat{H}$ are
$$
\langle\alpha j m|\hat{H}| \alpha j m\rangle=\langle\alpha j\|\hat{H}\| \alpha j\rangle
$$
The Wigner-Eckart theorem for scalar operators, Equation, to be written in the form
$$
\left\langle\alpha^{\prime \prime} j^{\prime \prime} m^{\prime \prime}|\hat{K}| \alpha^{\prime} j^{\prime} m^{\prime}\right\rangle=\left\langle\alpha^{\prime \prime} j^{\prime \prime}|| \hat{K} \| \alpha^{\prime} j^{\prime}\right\rangle\left\langle 00 ; j^{\prime} m^{\prime} \mid j^{\prime \prime} m^{\prime \prime}\right\rangle .
$$
The reason for writing the Wigner-Eckart theorem in this way will become clear once we have also considered the equivalent result for vector operators.
\subsection{consequence of Wigner-Eckart theorem (Vector Operators)}
Then the Wigner-Eckart theorem states that the matrix elements of $\hat{\boldsymbol{V}}$ between eigenstates $\left|\alpha^{\prime} j^{\prime} m^{\prime}\right\rangle$ and $\left|\alpha^{\prime \prime} j^{\prime \prime} m^{\prime \prime}\right\rangle$ of $\hat{\boldsymbol{J}}$ must be of the form
$$
\left\langle\alpha^{\prime \prime} j^{\prime \prime} m^{\prime \prime}\left|\hat{V}_m\right| \alpha^{\prime} j^{\prime} m^{\prime}\right\rangle=\left\langle\alpha^{\prime \prime} j^{\prime \prime}|| \hat{\boldsymbol{V}} \| \alpha^{\prime} j^{\prime}\right\rangle\left\langle 1 m ; j^{\prime} m^{\prime} \mid j^{\prime \prime} m^{\prime \prime}\right\rangle,
$$
where the final factor, $\left\langle 1 m ; j^{\prime} m^{\prime} \mid j^{\prime \prime} m^{\prime \prime}\right\rangle$, is a Clebsch-Gordan coefficient which can be obtained by considering the angular momentum combination $j^{\prime \prime}=1 \otimes j^{\prime}=j^{\prime}, j^{\prime} \pm 1$
\section{Electromagnetism}
\begin{definition}
    {cyclotron frequency}
    {$\omega_c= \dfrac{qB}{m}$}
    {
    This is the frequency with which particles moving transverse to a magnetic filed $B$ undergo circular orbit.\\
    $\omega_c$ should be really close to the spin precesssion frequency $\omega_s$.
    }
\end{definition}
\subsection{Aharanov-Bohm Effect}
\subsection{Gauge Invariance}
\subsubsection{Couloumb Gauge}
\subsubsection{Symmetric Gauge}
\subsection{Orbital Magnetic moment}
In Hamiltonian, the $L\cdot B$ term can be written as:
$\hat{H} = - \hat{\mu}_L\cdot B$
\begin{definition}
    {Orbital magnetic moment operator}
    {$- \hat{\mu}_L = \dfrac{q}{2m}\hat{L} \gamma_L$}
    {...}
\end{definition}
\begin{definition}
    {Gyromagnetic ratio, $\gamma_L$}
    {$\gamma_L = \dfrac{q}{2m}$}
    {...}
\end{definition}
For an electron (q=-e), the orbital magnetic moment operator is 
\subsection{Magnetic Moments}
\subsubsection{Electron}
\subsubsection{Muon}
\subsubsection{p, n, nuclei}
\subsection{Spin}
\subsubsection{Particle magnetic moment: spin-half}
\subsubsection{Spin Precession}
\subsubsection{Spin-half}
\subsubsection{Energy Eigenstates}


\subsubsection{Wave-function Evolution}
\subsection{Stern-Gerlach}
\subsection{Landau Levels}
\subsubsection{Landau Gauge}
\begin{example}
    {2D Electron Gas}
    {}
\end{example}
\section{Real Hydrogen Atom}
\subsection{Relativistic Corrections}
\subsection{Fine Structure}
\subsection{Hyperfine Structure}
\section{Symmetries}

\subsection{Symmetry Transformation}
\subsubsection{Time translation}
Take way: The time-dependent Shrodinger equation is a consequence of the invariance under time transformation
\subsection{The Wigner-Eckart Theorem(selection rule)}
\begin{theorem}
    {Wigner-Eckart Theorem}
    {$ \bra {\alpha'' j''m''}\hat{K}\ket{\alpha' j'm'} = \bra{\alpha''j''}|\hat{K}|\ket{\alpha'j'}$}
\end{theorem}
A particular case is that $ \bra {\alpha jm}\hat{K}\ket{\alpha jm} = \bra{\alpha j}|\hat{K}|\ket{\alpha j}$\\
i.e. the expectation values of a scalr opertator are independent of m and are given by the appropriate reduced matrix element of K
\subsection{Combining magnetic moment}
\section{Identical Particles}
Identical particles are indistinguishable.
\begin{enumerate}
    \item The normalized two-particle wavefunction $\psi\left(x_1, x_2\right)$, which gives the probability $\left|\psi\left(x_1, x_2\right)\right|^2 d x_1 d x_2$ of finding simultaneously one particle in the interval $x_1$ to $x_1+d x_1$ and another between $x_2$ to $x_2+d x_2$,
    \item only makes sense if $\left|\psi\left(x_1, x_2\right)\right|^2=\left|\psi\left(x_2, x_1\right)\right|^2$, since we can't know which of the two indistinguishable particles we are finding where.
    \item It follows from this that the wavefunction can exhibit two (and, generically, only two) possible symmetries under exchange: $\psi\left(x_1, x_2\right)=\psi\left(x_2, x_1\right)$ or $\psi\left(x_1, x_2\right)=-\psi\left(x_2, x_1\right) .^2$ 
    \item f two identical particles have a symmetric wavefunction in some state, particles of that type always have symmetric wavefunctions, and are called bosons.Similarly, particles having antisymmetric wavefunctions are called fermions. 
    \item We could achieve the necessary antisymmetrization for particles 1 and 2 by subtracting the same product wavefunction with the particles 1 and 2 interchanged, i.e. $\psi_a(1) \psi_b(2) \psi_c(3) \mapsto\left(\psi_a(1) \psi_b(2)-\psi_a(2) \psi_b(1)\right) \psi_c(3)$, ignoring the overall normalization for now. Such a sum over permutations is precisely the definition of the determinant. So, with the appropriate normalization factor:
    $$
    \psi_{a b c}(1,2,3)=\frac{1}{\sqrt{3 !}}\left|\begin{array}{lll}
    \psi_a(1) & \psi_b(1) & \psi_c(1) \\
    \psi_a(2) & \psi_b(2) & \psi_c(2) \\
    \psi_a(3) & \psi_b(3) & \psi_c(3)
    \end{array}\right|
    $$
\end{enumerate}
\subsection{Space and spin}
\subsection{Spin and statistics(fermions and bosons)}
\subsection{Exchange forces}
\subsection{The Helium atom}
\section{Multi-electron atoms}
\subsection{Periodic table}
\subsection{LS coupling(Hund's rule)}
\subsection{jj coupling}
\section{Zeeman effect/Stark effect/Molecules: $H^+_2$ and $H_2$}
\section{Time-dependent perturbation}
\section{Fermi Golden Rule}

\end{document}


