\documentclass[12pt,a4paper]{article}
\author{Xinyu Zhong\\Wolfson College}
\usepackage{physics, amsmath}
\usepackage{xcolor}
\usepackage[margin=0.5in]{geometry}

\title{Advanced Quantum Physics Notes}
%\date{2nd Nov 2021}

\newcommand{\definition}[3]
    {
    \textit{Definition #1: }
    \begin{center}
        {#2}
    \end{center}
    {#3}\\
    }
\newcommand{\theorem}[2]{\textbf{\textcolor{red}{#1: }}{#2}\\}
\newcommand{\example}[2]{\textbf{Example: #1}\\\textcolor{blue}{#2}\\}

\begin{document}

\begin{titlepage}
    \maketitle
\end{titlepage}

\tableofcontents

\newpage

\begin{abstract}
\noindent
Abstract of this course
\end{abstract}

\section{Revision}
\section{Perturbation Theory}
\subsection{Time-Independent Perturbation Theory}
\subsection{First-order Perturbation Theory}
\subsection{Second-order Perturbation Theory}
\begin{example}
    {Infinite square well with central bump}
    {...}
\end{example}
\begin{example}
    {Infinite square well in an electric field}
    {...}
\end{example}
\begin{example}
    {Harmonic Oscillator + Linear perturbation}
    {...}
\end{example}
\begin{example}
    {Van der Waals Interaction}
    {...}
\end{example}
\subsection{Degenerate Perturbation Theory}
\begin{example}
    {Perturbed 2D infinite square well}
    {...}
\end{example}
\subsection{Variation Method}
\begin{example}
    {Hydrogen atom ground state energy}
    {...}
\end{example}
\subsubsection{Rayleigh-Ritz Method}
\begin{example}
    {Hydrogen atom with finite proton mass}
    {...}
\end{example}
\section{Electromagnetism}
\subsection{Aharanov-Bohm Effect}
\subsection{Gauge Invariance}
\subsubsection{Couloumb Gauge}
\subsubsection{Symmetric Gauge}
\subsection{Orbital Magnetic moment}
In Hamiltonian, the $L\cdot B$ term can be written as:
$\hat{H} = - \hat{\mu}_L\cdot B$
\begin{definition}
    {Orbital magnetic moment operator}
    {$- \hat{\mu}_L = \dfrac{q}{2m}\hat{L} \gamma_L$}
    {...}
\end{definition}
\begin{definition}
    {Gyromagnetic ratio, $\gamma_L$}
    {$\gamma_L = \dfrac{q}{2m}$}
    {...}
\end{definition}
For an electron (q=-e), the orbital magnetic moment operator is 
\subsection{Magnetic Moments}
\subsubsection{Electron}
\subsubsection{Muon}
\subsubsection{p, n, nuclei}
\subsection{Spin}
\subsubsection{Particle magnetic moment: spin-half}
\subsubsection{Spin Precession}
\subsubsection{Spin-half}
\subsubsection{Energy Eigenstates}
\subsubsection{Wave-function Evolution}
\subsection{Stern-Gerlach}
\subsection{Landau Levels}
\subsubsection{Landau Gauge}
\begin{example}
    {2D Electron Gas}
    {}
\end{example}
\section{Real Hydrogen Atom}
\subsection{Relativistic Corrections}
\subsection{Fine Structure}
\subsection{Hyperfine Structure}
\section{Symmetries}

\subsection{Symmetry Transformation}
\subsection{The Wigner-Eckart Theorem(selection rule)}
\subsection{Combining magnetic moment}
\section{Identical Particles}
\subsection{Spin and statistics(fermions and bosons)}
\subsection{Exchange forces}
\subsection{The Helium atom}
\section{Multi-electron atoms}
\subsection{Periodic table}
\subsection{LS coupling(Hund's rule)}
\subsection{jj coupling}
\section{Zeeman effect/Stark effect/Molecules: $H^+_2$ and $H_2$}
\end{document}


