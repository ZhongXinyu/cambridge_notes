\documentclass[12pt,a4paper]{article}
\author{Xinyu Zhong\\Wolfson College}
\usepackage{physics, amsmath}
\usepackage{xcolor}
\usepackage[margin=0.5in]{geometry}
\title{Electrodynamics and Optics Notes}
%\date{2nd Nov 2021}

\usepackage{fancyhdr}
\pagestyle{fancyplain}
\fancyhf{}
\lhead{\fancyplain{}{Zachary Zhong, xz447@cam.ac.uk}}
\rhead{\fancyplain{}{Thermodynamics}}
\cfoot{\fancyplain{}{\thepage{}}}
\setlength {\headheight}{15pt}

\newcommand{\definition}[1]{\textcolor{red}{#1}{}}
\newcommand{\theorem}[2]{\textbf{\textcolor{red}{#1: }}\textcolor{red}{#2}}
\newcommand{\example}[1]{\par\textbf{Example: }\textcolor{blue}{#1}}
\begin{document}
\begin{titlepage}
    \maketitle
\end{titlepage}

\tableofcontents

\newpage

\begin{abstract}
\noindent
Abstract of this course
\end{abstract}

\section{Revision}
\section{Optics}
\subsection{Jones's Notation}
\subsection{Birefringent Material}
For isotropic medium :
$\vb{P}=\epsilon_0\chi\vb{E}$
$\vb{D} =\epsilon\epsilon_0\vb{E}$

\subsubsection{Discussion 1}
    \textit{Q: How an uniaxial birefringent material can be used to make a quarter wave plate.?}\\
    \begin{quote}
    \textcolor{blue}{
    Uniaxial birefringent material have principle refractive indices $n_o$, $n_o$ and $n_e$. 
    We can consider a plane-polarised EM wave $e^{i(kz-wt)}$ travels along $O_z$
    at a different speeds $c/n_f$ or $c/n_s$ depending on whether $\vb{E}$ is parallel to $O_x$ or $O_y$
    As the wave trasverse the plate, the phase will shift: $e^{ik(z=0)}\rightarrow e^{ik_f(z=d)}$,
    where $k_f=\dfrac{\omega n_f}{c}$.\\}
    So the phase shift will depend on the optical thickness,$d$ and also the refractive index:\\
    Along fast axis, the change is $e^{i\omega n_f d/c}$.\\
    Along slow axis, the change is $e^{i\omega n_s d/cs}$.\\
    The Jones matrix for the plate can be written as\\
    A quarter-wave plate is on with difference in phase shift corresponding to $\lambda /4 $\\
    \end {quote}

\section{Coherence}
\subsection{Coherence and Interference}
Interference ideas provide a useful quantified description for the degree of correlation, or degree of coherence.
\subsection{Temporal Coherence}
\subsubsection{Power Specturm of Temporal Coherence}
\theorem{Wiener Khinchine Theroem}
    {The FT of the auto correlation function of a funciton is the quare modulus of the FT of the function itself}
\subsection{Spatial Coherence}
\section{Electrodynamics}
\subsection{Gauge in EM}
\subsection{$\vb{A}$ in simple cases}
%Lecture 11:
\subsection{$\vb{A}$ in quantum mechanics}
\subsubsection{Hamiltonian}
\subsubsection{Aharanov-Bohm Effect}
\subsection {Maxwell Equation in terms of A and $\phi$}
$\dfrac{\epsilon\mu}{c^2}\ddot{\phi}-\nabla^2\phi = \dfrac{\rho}{\epsilon\epsilon_0}$\\
$\dfrac{\epsilon\mu}{c^2}\ddot{\vb{A}}-\nabla^2\vb{A} = \mu\mu_0\vb{J}$
\subsubsection {Lorenz condition}
A suitable gauge that simplifies maxwell equation, by choosing the gauge:
$\div \vb{A}+ \epsilon\mu \dfrac{\dot{\phi}}{c^2} = 0$\\
Note: for static filed the Lorenz condition is chosen to be Coulomb gauge.
\subsection {Solution for A and $\phi$}
Square bracket mean "evaluated at the \textit{retarded time}" $t-\frac{|\vb{r}=\vb{r'}|}{c}$
\section{Dipole Radiation}
Accelerating charges are of the interest of radiating.
\subsection{The Hertzian Dipole}
%Lecture 12:

    
    
\end{document}
