\documentclass[12pt,a4paper]{article}

\usepackage{import}
\import{../Template/}{format.tex}

\newcommand{\topic}{Astrofluid Dynamics}

\begin{document}

\begin{titlepage}
    \maketitle
\end{titlepage}

\tableofcontents

\newpage

\begin{abstract}
\noindent
Abstract of this course
\end{abstract}
\section{Some basic concepts}
Collisional v collisionless fluids\\
Eulerian and Lagrangian framework\\
Concepts of streamlines, particle paths and streaklines:\\
They coinside if the flow is steady.i.e. \\
\section{Formulation of the Fluid Equations}
This chapter talked about the conservation of mass and momentum. Some of spec
\subsection{Conservation of mass}
\subsection{Conservation of momentum}
We consider 4 different which contribute to the change of momentum,?
\section{Gravitation}
In this section, 
we used $\va{g}$ to denote gravitational acceleration; 
$\Psi$ to denote gravitational potential;
and $\Omega$ to denote the energy required to take the system of point masses to infinity\\
\\
\example{Spherical distribution of mass}{}
\example{Infinitely cylindrical symmetrical mass}{}
\example{Infinite planar distribution of masses}{}
\example{Finite axisymmetric disk}{}
\subsection{Potential of a Spherical Mass Distribution}
$\Psi$ is affected by any matter outside r through our choice of setting $Psi$ at infinity. 
i.e. We \textbf{can't} say that $\Psi = -GM/r$

\subsection{Gravitational Potential Energy}
\subsection{Virial Theorem}
\theorem{Virial Theorem}{ states that for a system in steady state, $I\equiv mr^2 =constant $, $2T+\Omega =0 $}
Kinetic energy T has contribution from local flows and random/thermal motions.\\
A result of virial theorem is that the gravitational potential sets the temperature or velocity dispersion of the system.
\section{Equation of state and the energy equation}
So far we have 4 variables, density $\rho$, pressure,$p$, gravitational potential $\Psi$ and velocity which is $\vb*{v}$.
In terms of equations to solve them, we have the scalar equation of mass conservation and the the vector equation for conservation of momentum. We also have Poisson equation for the potential term.
Now what we need is another equation: `Equation of state' to determinate pressure (which is a thermodynamic property). While doing so we might Introduce another equation, the energy equation when the system is not barotropic,i.e. $p$ is a function of $T$.
\subsection{Equation of state}
    \begin{enumerate}
        \item Astrophysical fluid are treated as ideal gas, and corresponding EoS is:
        \begin{equation}
            p=nk_B T =\frac{k_B}{\mu m_p}\rho T
        \end{equation}
        where $\rho$ is the mean particle mass.
        \item This EoS introduces another scalar field, temperature, $T$
    \end{enumerate}
    \subsubsection{Barotropic}
        Barotropic means that $p$ independent of $T$. i.e. only a fucntion of $\rho$. This comes in two cases: Isothermal and Adiabatic.
        \begin{enumerate}
            \item Isothermal: Constant T
            \item Adiabatic: Ideal has undergoes reversible thermodynamic changes
            \begin{equation}
                p= K \rho^{\gamma}
            \end{equation} 
            \item Derivation of $C_p$ and $C_v$ for both cases.
        \end{enumerate}
    \subsubsection{Energy equation for non-Barotropic case}
        \begin{enumerate}
            \item It starts from first law
            \begin{equation}
                \bar{d}Q = d \mathcal{E} + pdV
            \end{equation}
            \item total energy per unit volume is:
            \begin{equation}
                E=\rho(\frac{1}{2}u^2 +\Psi + \mathcal{E})
            \end{equation}
            \item take material derivative, note that $\dfrac{DE}{Dt}=\pdv{E}{t}+\vb{u}\dotproduct\grad{E}$, gives Energy Equation
            \begin{equation}
                \pdv{E}{t}+\div{[(E+p)\vb{u}]= \rho\pdv{\Psi}{t}-\rho \dot{Q}_{cool}}
            \end{equation}
            In many settings $\partial{\Psi}/\partial{\Psi}=0$, If there is no cooling, the equation expresses the conservation for energy in which
            the total energy density $E$ is driven by the divergence of the enthalpy flux $(E+p)\vb{u}$
        \end{enumerate}
        

\end{document}