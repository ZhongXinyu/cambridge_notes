\documentclass[12pt,a4paper]{article}

\usepackage{import}
\import{../Template/}{format.tex}

\newcommand{\topic}{Astrofluid Dynamics}

\begin{document}

\begin{titlepage}
    \maketitle
\end{titlepage}

\tableofcontents

\newpage

\begin{abstract}
\noindent
Abstract of this course
\end{abstract}
\section{Some basic concepts}
Collisional v collisionless fluids\\
Eulerian and Lagrangian framework\\
Concepts of streamlines, particle paths and streaklines:\\
They coinside if the flow is steady.i.e. \\
\section{Formulation of the Fluid Equations}
This chapter talked about the conservation of mass and momentum.
\subsection{Conservation of mass}
\begin{equation}
    \pdv{\rho}{t} + \div{\rho \vb{u}} = 0
\end{equation}
\subsection{Conservation of momentum}
\begin{equation}
    \rho \pdv{\vb{u}}{t} + \rho (\vb{u}\dotproduct \grad)vb{u} = -\grad{p}+ \rho \vb{g}
\end{equation}
We consider 4 different which contribute to the change of momentum?
\section{Gravitation}
In this section, 
we used $\va{g}$ to denote gravitational acceleration; 
$\Psi$ to denote gravitational potential;
and $\Omega$ to denote the energy required to take the system of point masses to infinity\\
\\
\example{Spherical distribution of mass}{}
\example{Infinitely cylindrical symmetrical mass}{}
\example{Infinite planar distribution of masses}{}
\example{Finite axisymmetric disk}{}
\subsection{Potential of a Spherical Mass Distribution}
$\Psi$ is affected by any matter outside r through our choice of setting $Psi$ at infinity. 
i.e. We \textbf{can't} say that $\Psi = -GM/r$

\subsection{Gravitational Potential Energy}
\subsection{Virial Theorem}
\theorem{Virial Theorem}{ states that for a system in steady state, $I\equiv mr^2 =constant $, $2T+\Omega =0 $}
Kinetic energy T has contribution from local flows and random/thermal motions.\\
A result of virial theorem is that the gravitational potential sets the temperature or velocity dispersion of the system.
\section{Equation of state and the energy equation}
So far we have 4 variables, density $\rho$, pressure,$p$, gravitational potential $\Psi$ and velocity which is $\vb*{v}$.
In terms of equations to solve them, we have the scalar equation of mass conservation and the vector equation for conservation of momentum. We also have Poisson equation for the potential term.
Now what we need is another equation: `Equation of state' to determinate pressure (which is a thermodynamic property). While doing so we might Introduce another equation, the energy equation when the system is not barotropic, i.e. $p$ is a function of $T$.
\subsection{Equation of state}
    \begin{enumerate}
        \item Astrophysical fluid are treated as ideal gas, and corresponding EoS is:
        \begin{equation}
            p=nk_B T =\frac{k_B}{\mu m_p}\rho T
        \end{equation}
        where $\rho$ is the mean particle mass.
        \item This EoS introduces another scalar field, temperature, $T$
    \end{enumerate}
    \subsubsection{Barotropic}
        Barotropic means that $p$ independent of $T$.\ i.e. only a function of $\rho$. This comes in two cases: Isothermal and Adiabatic.
        \begin{enumerate}
            \item Isothermal: Constant T
            \item Adiabatic: Ideal has undergoes reversible thermodynamic changes
            \begin{equation}
                p= K \rho^{\gamma}
            \end{equation} 
            \item Derivation of $C_p$ and $C_v$ for both cases.
        \end{enumerate}
    \subsubsection{Energy equation for non-Barotropic case}
        \begin{enumerate}
            \item It starts from first law
            \begin{equation}
                \bar{d}Q = d \mathcal{E} + pdV
            \end{equation}
            \item total energy per unit volume is:
            \begin{equation}
                E=\rho(\frac{1}{2}u^2 +\Psi + \mathcal{E})
            \end{equation}
            \item take material derivative, note that $\frac{DE}{Dt}=\pdv{E}{t}+\vb{u}\dotproduct\grad{E}$, gives Energy Equation
            \begin{equation}
                \pdv{E}{t}+\div{[(E+p)\vb{u}]= \rho\pdv{\Psi}{t}-\rho \dot{Q}_{cool}}
            \end{equation}
            In many settings $\partial{\Psi}/\partial{\Psi}=0$, If there is no cooling, the equation expresses the conservation for energy in which
            the total energy density $E$ is driven by the divergence of the enthalpy flux $(E+p)\vb{u}$
        \end{enumerate}
    \subsection{Heating and Cooling Processes}
        Combining cooling and heating effects, we can parametrise $\dot{Q}_{cool}$
        \begin{equation}
            \dot{Q}_{cool} = A \rho T^{\alpha} - H
        \end{equation}
        where the first and second term on RHS means radiative cooling and cosmic ray heating
    \subsection{Energy Transport process}
        Transport processes move energy through the fluid, via
        Thermal conduction, convention and radiation transport.
\section{A lot of different system}
    \subsection{Full set of equations describing the dynamics of an ideal non-relativisitc fluid}
        \begin{align*}
            \pdv{\rho}{t} + \div{\rho \vb{u}} = 0 \tag{Continuity Equation}\\
            \rho \pdv{\vb{u}}{t}+\rho (\vb{u}\dotproduct \div)\vb{u}=-\div{p}-\rho\div{\Psi}\tag{Momentum Equation}\\
            \laplacian{\Psi} = 4\pi\rho \tag{Poisson's Equation}\\
        \end{align*}
        \begin{equation}
            \tag{Energy Equation}
        \end{equation}
        \begin{equation}
            \tag{Definition fo Total energy}
        \end{equation}
        \begin{equation}
            \tag{EoS of total energy}
        \end{equation}
        \begin{equation}
            \tag{Internal Energy}
        \end{equation}
    \subsection{Hydrostatic Equilibrium}
        A System of hydrostatic equilibrium if 
        \begin{equation}
            \vb{u} = \pdv{}{t} =0
        \end{equation}
        Continuity equation is trivially satisfied
        Sub into momentum equation gives:
        \begin{equation}
            \frac{1}{\rho}\grad{p} = - \grad{\Psi}\tag{Equation of Hydrostatic equilibirum}
        \end{equation}
        \begin{example}
            {Isothermal atmosphere}{
                Isothermal atmosphere with constant $\vb{g}=-g\hat{\vb{z}}$
                \begin{equation}
                    \rho = \rho_0 \exp{-\frac{\mu g} {R_{*} T} z}
                \end{equation}
                i.e. exponential atmosphere
            }
        \end{example}
        \begin{example}
            {Isothermal self-gravitating slab}{
                Isothermal atmosphere with constant $\vb{g}=-g\hat{\vb{z}}$
                \begin{equation}
                    \laplacian{\Psi} = 4\pi G\rho
                \end{equation}
                Is gives
                \begin{equation}
                    \Psi-\Psi_0 = 2A\ln{\cosh{\sqrt{\frac{2\pi G\rho}{A}z}}}
                \end{equation}
                \begin{equation}
                    \rho = \frac{\rho_0}{\cosh^2{\sqrt{\frac{2\pi G\rho}{A}z}}}
                \end{equation}
            }
        \end{example}
    \subsection{Stars/Self-Gravitating Polytropes}
    \begin{enumerate}
        \item In this section, we consider spherically-symmetric, self-gravitating system in hydrostatic equilibirum, which is called "star". Note that non-rotating stars are barotropes.
        \item In this system, we have $p=p(\rho)$,
        and barotropic EoS can be written as $p=K\rho^{1+1/n}$
        If the star is isentropic, i.e. constant entropy. ${1+1/n} = \gamma$. 
        \item When $n$ is constant, the structure is called a Polytrope
        \item Assuming a polytropics EoS, the equation of hydrostatic equilibrium is 
        \begin{align*}
            -\grad{\Psi} = \frac{1}{\rho}\grad(K\rho^{1+1/n}) = (n+1)\grad({K\rho^{1/n}})
            \intertext{Which has a solution for $\rho$:}
            \rho  =\left(\frac{\Psi_T-\Psi}{(n+1)K}\right)^n
            \intertext{In which $\Psi_T$ is $\Psi(\rho=0)$ at surface.}
            \intertext{
            Given boundary condition at for central density and central potential $\rho_c$ and $\Psi_c$.}
            \rho =\rho_c\left(\frac{\Psi_T-\Psi}{\Psi_T-\Psi_C}\right)^n
            \intertext{Now feed into Poisson Equation gives a differential equation for $\Psi$}
            \intertext{Define $\theta$ (which is equivalent to $\Psi$)and $\mathcal{E}$ (which is scaled radial coordinate), we get Lane-Emden Eqn of index n}
            \frac{1}{\mathcal{E}}\frac{d}{d\mathcal{E}^2}\left(\mathcal{E}^2\dv{\theta}{\mathcal{E}}\right)=-\theta^n
            \intertext{which is an differential equation for $\theta$.}
            \intertext{Analytical solution for n =0,1,5}
        \end{align*}
    \end{enumerate}
    \subsection{Isothermal Sphere}
        \begin{enumerate}
        \item Isothemal case means that $p=K\rho$, 
        \begin{align}
            \intertext{From the momentum equation gives:}
            \Psi - \Psi_c = -K\ln(\rho/\rho_c)
            \intertext{From Possion's Equation gives differential equation for $\rho$}
            \intertext{Take $\rho = \rho_c e^{-\Psi}$,we get a differential equation for $\Psi$}
            \frac{1}{\mathcal{E}}\frac{d}{d\mathcal{E}^2}\left(\mathcal{E}^2\dv{\Psi}{\mathcal{E}}\right)=e^{-\Psi}
            \intertext{which is equivalent to Lane-Equation in this isothermal case.}
        \end{align}
    \end{enumerate}
    \subsection{Scaling Relations}
        Scaling relation relates the mass and radius of a polytrope star.
\section{Sound waves, supersonic flows and shock waves}
\subsection{Sound Waves}
    In this section, we find the sound wave a derivative of a $p(\rho)$
    \begin{example}
        {Isothermal}{$c^2_s=\sqrt{\frac{R_{*}T}{\mu}}$}
    \end{example}
    \begin{example}
        {Adiabatic}{$c^2_s=\sqrt{\mu\frac{R_{*}T}{\mu}}$}
    \end{example}
\subsection{Sound Waves in Stratified Atmosphere}
    Now we consider sound waves traveling in a fluid with a background structure, i.e. constant gravitational field.
    We will end up with a dispersion relationship and a $k$ that could be imaginary, meaning a decaying wave.
    $$
    \omega^2=c_u^2\left(k^2-\frac{\mathrm{i} k}{H}\right)
    $$
    DISPERSION RELATION
\subsection{Sound waves across interfaces}
    We then matches B.C. to get reflection coefficient and transmission radiation.
    $$
    t=\frac{2 k_1}{k_1+k_2} \quad r=\frac{k_1-k_2}{k_1+k_2}
    $$
\subsection{Supersonic Fluid and Shocks}
    Shock occurs when there is a disturbance in the fluid cause by compression by a large factor.\\
    \begin{definition}
        {Mach Number}{$M_1= u_1/c_{s,1}$}{This number helps us to simplify the R-H relationships in the next section.}
    \end{definition}
\subsection{The Rankine-Hugoniot Relations}
    We discovered 3 R-H relations, from conservation of mass, momentum and energy.
    \begin{align*}
        \rho_1 u_1&=\rho_2 u_2\\
        \rho_1 u_1^2+p_1&=\rho_2 u_2^2+p_2\\
        \frac{1}{2} u_1^2+\mathcal{E}_1+\frac{p_1}{\rho_1}&=\frac{1}{2} u_2^2+\mathcal{E}_2+\frac{p_2}{\rho_2}
    \end{align*}
    \subsubsection{Adiabatic Case}
    The last R-H relationship becomes, in an adiabatic idea gas:
    \begin{align*}
        \frac{1}{2} u_1^2+\frac{c_{s, 1}^2}{\gamma-1}&=\frac{1}{2} u_2^2+\frac{c_{s, 2}^2}{\gamma-1}
    \end{align*}

    \subsubsection{Isothermal Case}
    The last R-H relationship becomes $T_1=T_2$
    with this relationship, we have $c_s^2=u_1u_2$ and 
    $$ 
    \frac{\rho_2}{\rho_1}=\frac{u_1}{u_2}=\left(\frac{u_1}{c_s}\right)^2=M_1^2
    $$
    Note that if $u_1>c_s$ then $u_2<c_s$, flow behind the shock is subsonic.

\section{Bernuolli's Equation}
This gives us Bernoulli's Principle: For steady barotropic flows, the quantity
$$
H=\frac{1}{2} u^2+\int \frac{\mathrm{d} p}{\rho}+\Psi
$$
is constant along a streamline. The quantity $H$ is called Bernoulli's constant.\\
If $p=0, H=$ constant is the statement that kinetic $+$ potential energy is constant along streamlines.\\

If $p \neq 0$, pressure differences accelerate or decelerate the flow as it flows along the streamline.
\subsection{Rotational and irrotational Flow}
Let's start with the momentum equation:
$$
\frac{\partial \mathbf{u}}{\partial t}+(\mathbf{u} \cdot \boldsymbol{\nabla}) \mathbf{u}=-\frac{1}{\rho} \boldsymbol{\nabla} p-\nabla \Psi
$$


\subsection{Laplacian}
Laplacian Equation is derived if we have incompressible and irrotational flow. i.e $\div{\vb{u}=0}$ and $\curl{\vb{u}}=0$
\end{document}