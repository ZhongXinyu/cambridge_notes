\documentclass[10pt,a4paper]{article}
\author{Xinyu Zhong\\Wolfson College}
\usepackage{physics, amsmath}
\usepackage{xcolor}
\usepackage[margin=0.5in]{geometry}
\usepackage{fancyhdr}
\pagestyle{fancyplain}
\fancyhf{}


%\date{2nd Nov 2021}
\lhead{\fancyplain{}{Zachary Zhong, xz447@cam.ac.uk}}
\rhead{\fancyplain{}{Thermodynamics}}
\cfoot{\fancyplain{}{\thepage{}}}
\setlength{\headheight}{15pt}

\newcommand{\definition}[3]
    {
    \textit{Definition #1: }
    \begin{center}
        {#2}
    \end{center}
    {#3}\\
    }
\newcommand{\theorem}[2]{\textbf{\textcolor{red}{#1: }}{#2}\\ \\}
\newcommand{\example}[2]{\textbf{Example: #1}\\\textcolor{blue}{#2}\\ \\}
\title{Notes}

\begin{document}

\begin{titlepage}
    \maketitle
\end{titlepage}

\tableofcontents

\newpage

\begin{abstract}
\noindent
Abstract of this course
\end{abstract}
\section{Some basic concepts}
Collisional v collisionless fluids\\
Eulerian and Lagrangian framework\\
Concepts of streamlines, particle paths and streaklines:\\
They coinside if the flow is steady.i.e. \\
\section{Formulation of the Fluid Equations}
This chapter talked about the conservation of mass and momentum. Some of spec
\subsection{Conservation of mass}
\subsection{Conservation of momentum}
We consider 4 different which contribute to the change of momentum,?
\section{Gravitation}
In this section, 
we used $\va{g}$ to denote gravitational acceleration; 
$\Psi$ to denote gravitational potential;
and $\Omega$ to denote the energy required to take the system of point masses to infinity\\
\\
\example{Spherical distribution of mass}{}
\example{Infinitely cylindrical symmetrical mass}{}
\example{Infinite planar distribution of masses}{}
\example{Finite axisymmetric disk}{}
\subsection{Potential of a Spherical Mass Distribution}
$\Psi$ is affected by any matter outside r through our choice of setting $Psi$ at infinity. 
i.e. We \textbf{can't} say that $\Psi = -GM/r$

\subsection{Gravitational Potential Energy}
\subsection{Virial Theorem}
\theorem{Virial Theorem}{ states that for a system in steady state, $I\equiv mr^2 =constant $, $2T+\Omega =0 $}
Kinetic energy T has contribution from local flows and random/thermal motions.\\
A result of virial theorem is that the gravitational potential sets the temperature or velocity dispersion of the system.

\end{document}