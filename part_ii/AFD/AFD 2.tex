\documentclass[12pt,a4paper]{article}

\usepackage{import}
\import{../Template/}{format.tex}

\newcommand{\topic}{Astrofluid Dynamics}

\begin{document}

\begin{titlepage}
    \maketitle
\end{titlepage}

\tableofcontents

\newpage

\begin{abstract}
\noindent
Abstract of this course
\end{abstract}
\section{Some basic concepts}
Collisional v collisionless fluids\\
Eulerian and Lagrangian framework\\
Concepts of streamlines, particle paths and streaklines:\\
They coincide if the flow is steady.i.e. \\
\section{Formulation of the Fluid Equations}
This chapter talked about the conservation of mass and momentum.
\subsection{Conservation of mass}
\begin{equation}
    \pdv{\rho}{t} + \div{\rho \vb{u}} = 0
\end{equation}
\subsection{Conservation of momentum}
\begin{equation}
    \rho \pdv{\vb{u}}{t} + \rho (\vb{u}\dotproduct \grad)vb{u} = -\grad{p}+ \rho \vb{g}
\end{equation}
We consider 4 different which contribute to the change of momentum.
\section{Gravitation}
In this section, 
we used $\va{g}$ to denote gravitational acceleration; 
$\Psi$ to denote gravitational potential;
and $\Omega$ to denote the energy required to take the system of point masses to infinity\\
\\
\example{Spherical distribution of mass}{}
\example{Infinitely cylindrical symmetrical mass}{}
\example{Infinite planar distribution of masses}{}
\example{Finite axisymmetric disk}{}
\subsection{Potential of a Spherical Mass Distribution}
$\Psi$ is affected by any matter outside r through our choice of setting $Psi$ at infinity. 
i.e. We \textbf{can't} say that $\Psi = -GM/r$

\subsection{Gravitational Potential Energy}
\subsection{Virial Theorem}
\theorem{Virial Theorem}{ states that for a system in steady state, $I\equiv mr^2 =constant $, $2T+\Omega =0 $}
Kinetic energy $T$ has a contribution from local flows and random/thermal motions.\\
A result of the virial theorem is that the gravitational potential sets the temperature or velocity dispersion of the system.
\section{Equation of state and the energy equation}
So far we have 4 variables, density $\rho$, pressure,$p$, gravitational potential $\Psi$ and velocity which is $\vb*{v}$.
In terms of equations to solve them, we have the scalar equation of mass conservation and the vector equation for the conservation of momentum. We also have the Poisson equation for the potential term.
Now what we need is another equation: an `Equation of state' to determine pressure (which is a thermodynamic property). While doing so we might Introduce another equation, the energy equation when the system is not barotropic, i.e. $p$ is a function of $T$.
\subsection{Equation of state}
    \begin{enumerate}
        \item Astrophysical fluids are treated as an ideal gas, and the corresponding EoS is:
        \begin{equation}
            p=nk_B T =\frac{k_B}{\mu m_p}\rho T
        \end{equation}
        where $\rho$ is the mean particle mass.
        \item This EoS introduces another scalar field, temperature, $T$
    \end{enumerate}
    \subsubsection{Barotropic}
        Barotropic means that $p$ is independent of $T$.\ i.e. only a function of $\rho$. This comes in two cases: Isothermal and Adiabatic.
        \begin{enumerate}
            \item Isothermal: Constant T
            \item Adiabatic: Ideal has undergone reversible thermodynamic changes
            \begin{equation}
                p= K \rho^{\gamma}
            \end{equation} 
            \item Derivation of $C_p$ and $C_v$ for both cases.
        \end{enumerate}
    \subsubsection{Energy equation for non-Barotropic case}
        \begin{enumerate}
            \item It starts from the first law
            \begin{equation}
                \bar{d}Q = d \mathcal{E} + pdV
            \end{equation}
            \item total energy per unit volume is:
            \begin{equation}
                E=\rho(\frac{1}{2}u^2 +\Psi + \mathcal{E})
            \end{equation}
            \item take material derivative, note that $\frac{DE}{Dt}=\pdv{E}{t}+\vb{u}\dotproduct\grad{E}$, gives Energy Equation
            \begin{equation}
                \pdv{E}{t}+\div{[(E+p)\vb{u}]= \rho\pdv{\Psi}{t}-\rho \dot{Q}_{cool}}
            \end{equation}
            In many settings $\partial{\Psi}/\partial{\Psi}=0$, If there is no cooling, the equation expresses the conservation for energy in which
            the total energy density $E$ is driven by the divergence of the enthalpy flux $(E+p)\vb{u}$
        \end{enumerate}
    \subsection{Heating and Cooling Processes}
        Combining cooling and heating effects, we can parametrise $\dot{Q}_{cool}$
        \begin{equation}
            \dot{Q}_{cool} = A \rho T^{\alpha} - H
        \end{equation}
        where the first and second term on RHS means radiative cooling and cosmic ray heating
    \subsection{Energy Transport process}
        Transport processes move energy through the fluid, via
        Thermal conduction, convention and radiation transport.
\section{A lot of different systems}
    \subsection{Full set of equations describing the dynamics of an ideal non-relativisitc fluid}
        \begin{align*}
            \pdv{\rho}{t} + \div{\rho \vb{u}} = 0 \tag{Continuity Equation}\\
            \rho \pdv{\vb{u}}{t}+\rho (\vb{u}\dotproduct \div)\vb{u}=-\div{p}-\rho\div{\Psi}\tag{Momentum Equation}\\
            \laplacian{\Psi} = 4\pi\rho \tag{Poisson's Equation}\\
        \end{align*}
        \begin{equation}
            \tag{Energy Equation}
        \end{equation}
        \begin{equation}
            \tag{Definition fo Total energy}
        \end{equation}
        \begin{equation}
            \tag{EoS of total energy}
        \end{equation}
        \begin{equation}
            \tag{Internal Energy}
        \end{equation}
    \subsection{Hydrostatic Equilibrium}
        A System of hydrostatic equilibrium if 
        \begin{equation}
            \vb{u} = \pdv{}{t} =0
        \end{equation}
        The continuity equation is trivially satisfied\\        
        The momentum equation gives:
        \begin{equation}
            \frac{1}{\rho}\grad{p} = - \grad{\Psi}\tag{Equation of Hydrostatic equilibirum}
        \end{equation}
        \begin{example}
            {Isothermal atmosphere}{
                Isothermal atmosphere with constant $\vb{g}=-g\hat{\vb{z}}$
                \begin{equation}
                    \rho = \rho_0 \exp{-\frac{\mu g} {R_{*} T} z}
                \end{equation}
                i.e. exponential atmosphere
            }
        \end{example}

        \begin{example}
            {Isothermal self-gravitating slab}{
                Isothermal atmosphere with 
                \begin{equation}
                    \laplacian{\Psi} = 4\pi G\rho
                \end{equation}
                Is gives
                \begin{equation}
                    \Psi-\Psi_0 = 2A\ln{\cosh{\sqrt{\frac{2\pi G\rho}{A}z}}}
                \end{equation}
                \begin{equation}
                    \rho = \frac{\rho_0}{\cosh^2{\sqrt{\frac{2\pi G\rho}{A}z}}}
                \end{equation}
            }
        \end{example}
    \subsection{Stars/Self-Gravitating Polytropes(example of hydrostatic equilibrium)}
    \begin{enumerate}
        \item In this section, we consider a spherically-symmetric, self-gravitating system in hydrostatic equilibrium, which is called a "star". Note that non-rotating stars are barotropes.
        \item In this system, we have $p=p(\rho)$,
        and barotropic EoS can be written as $p=K\rho^{1+1/n}$
        If the star is isentropic, i.e. constant entropy. ${1+1/n} = \gamma$. 
        \item When $n$ is constant, the structure is called a Polytrope
        \item Assuming a polytropics EoS, the equation of hydrostatic equilibrium is 
        \begin{align*}
            -\grad{\Psi} = \frac{1}{\rho}\grad(K\rho^{1+1/n}) = (n+1)\grad({K\rho^{1/n}})
            \intertext{Which has a solution for $\rho$:}
            \rho  =\left(\frac{\Psi_T-\Psi}{(n+1)K}\right)^n
            \intertext{In which $\Psi_T$ is $\Psi(\rho=0)$ at surface.}
            \intertext{
            Given boundary condition for central density and central potential $\rho_c$ and $\Psi_c$.}
            \rho =\rho_c\left(\frac{\Psi_T-\Psi}{\Psi_T-\Psi_C}\right)^n
            \intertext{Now feed into Poisson Equation gives a differential equation for $\Psi$}
            \intertext{Define $\theta$ (which is equivalent to $\Psi$)and $\mathcal{E}$ (which is scaled radial coordinate), we get Lane-Emden Eqn of index n}
            \frac{1}{\mathcal{E}}\frac{d}{d\mathcal{E}^2}\left(\mathcal{E}^2\dv{\theta}{\mathcal{E}}\right)=-\theta^n
            \intertext{which is an differential equation for $\theta$.}
            \intertext{Analytical solution for n =0,1,5}
        \end{align*}
    \end{enumerate}
    \subsection{Isothermal Sphere(case where $n\approx \infty$)}
        \begin{enumerate}
        \item Isothermal case means that $p=K\rho$, 
        \begin{align}
            \intertext{From the momentum equation gives:}
            \Psi - \Psi_c = -K\ln(\rho/\rho_c)
            \intertext{From Possion's Equation gives a differential equation for $\rho$}
            & \nabla^2 \Psi=4 \pi G \rho \\
            \Rightarrow \quad & \frac{1}{r^2} \frac{\mathrm{d}}{\mathrm{d} r}\left(r^2 \frac{\mathrm{d} \Psi}{\mathrm{d} r}\right)=4 \pi G \rho \\
            \Rightarrow \quad & \frac{K}{r^2} \frac{\mathrm{d}}{\mathrm{d} r}\left(r^2 \frac{1}{\rho} \frac{\mathrm{d} \rho}{\mathrm{d} r}\right)=-4 \pi G \rho
            \intertext{Take $\rho = \rho_c e^{-\Psi}$,we get a differential equation for $\Psi$}
            \frac{1}{\mathcal{E}^2}\frac{d}{d\mathcal{E}}\left(\mathcal{E}^2\dv{\Psi}{\mathcal{E}}\right)=e^{-\Psi}
            \intertext{which is equivalent to Lane-Equation in this isothermal case.}
        \end{align}
    \end{enumerate}
    \subsection{Short summary for the cases above:}
        In these cases, we are solving the spatial dependence of $P$,$\rho$ and $\Psi$, they are "equivalent" to each other, as they are connected by the barotropic equation or Poison equation.

        In the Lane-Emden equation, we have a defined $\theta$ and $\mathcal{E}$, which are equivalent to $\Psi$ and $r$.

        $\theta$ satisfy boundary condition where at centre $\theta(0)=1$, $\dv{\theta}{\mathcal{E}}= 0 $and at boundary, $\theta({\mathcal{E}_{max}})=0$.
    \subsection{Scaling Relations}
        Scaling relation relates the mass and radius of a polytrope star.
        Stars usually have adiabatic relation, for example, a monotonic gas with $\gamma = 5/3$ i.e. $n = 3/2$

        For all the stars with the same $n$, they are differentiated by the central density $\rho_c$. 

        Thus, the mass and radius of a star are
        determined and $\rho_c$, and by eliminating $\rho_c$ we get the scaling relation.
        
\section{Sound waves, supersonic flows and shock waves}
\subsection{Sound Waves}
    We now start a discussion of how disturbances can propagate in a fluid. We begin by talking about sound waves in a uniform medium (no gravity). We proceed by conducting a first-order perturbation analysis of the fluid equations:
    $$
    \begin{aligned}
    & \frac{\partial \rho}{\partial t}+\boldsymbol{\nabla} \cdot(\rho \mathbf{u})=0 \\
    & \frac{\partial \mathbf{u}}{\partial t}+(\mathbf{u} \cdot \boldsymbol{\nabla}) \mathbf{u}=-\frac{1}{\rho} \boldsymbol{\nabla} p
    \end{aligned}
    $$
    The equilibrium around which we will perturb is
    $$
    \begin{array}{ll}
    \rho=\rho_0 & \text { (uniform \& constant) } \\
    p=p_0 & \text { (uniform \& constant) } \\
    \mathbf{u}=\mathbf{0} &
    \end{array}
    $$
    We consider small perturbations and write in Lagrangian terms (Lagrangian meaning the change of quantities are for a given fluid element)
    $$
    \begin{aligned}
    & p=p_0+\Delta p \\
    & \rho=\rho_0+\Delta \rho \\
    & \mathbf{u}=\Delta \mathbf{u}
    \end{aligned}
    $$

    Take them into the momentum and continuity equation, we get the wave equation:
    $$
    \frac{\partial^2(\Delta \rho)}{\partial t^2}=\left.\frac{\mathrm{d} p}{\mathrm{~d} \rho}\right|_{\rho=\rho_0} \nabla^2(\Delta \rho)
    $$

    In this section, we find the sound wave a derivative of a $p(\rho)$

    \begin{example}
        {Isothermal sound wave }
        {$c^2_s=\sqrt{\frac{R_{*}T}{\mu}}$}
    \end{example}

    \begin{example}
        {Adiabatic sound wave }
        {$c^2_s=\sqrt{\gamma\frac{R_{*}T}{\mu}}$}
    \end{example}

\subsection{Sound Waves in Stratified Atmosphere}
    Now we consider sound waves traveling in a fluid with a background structure, i.e. constant gravitational field. We now discover that the sound waves are dispersive, by doing the same operation as the previous section
    We will end up with a dispersion relationship and a $k$ that could be imaginary, meaning a decaying wave.
    $$
    \omega^2=c_u^2\left(k^2-\frac{\mathrm{i} k}{H}\right)
    $$
    or,
    $$
    k=\frac{\mathrm{i}}{2 H} \pm \sqrt{\frac{\omega^2}{c_u^2}-\frac{1}{4 H^2}}
    $$
    Then by discussing the magnitude of the frequency, we get complex $k$ or purely imaginary $k$
\subsection{Sound waves across interfaces}
    Across interfaces, we have:
    incident wave:
    $$
    e^{\mathrm{i}\left(k_1 x-\omega t\right)}
    $$
    reflected wave:
    $$
    r e^{\mathrm{i}\left(-k_1 x-\omega t\right)}
    $$
    transversed Wave
    $$
    t e^{\mathrm{i}\left(k_2 x-\omega t\right)}
    $$
    We then match B.C. to get the reflection coefficient and transmission radiation.
    $$
    t=\frac{2 k_1}{k_1+k_2} \quad r=\frac{k_1-k_2}{k_1+k_2}
    $$
\subsection{Supersonic Fluid and Shocks}
    Shock occurs when there is a disturbance in the fluid caused by compression by a large factor.\\
    The velocity of the disturbance relative to the observer, $v^{\prime}$, is the vector sum of the fluid velocity and the disturbance velocity relative to the fluid.
    
    - Subsonic case: $v^{\prime}$ sweeps $4 \pi$ steradians;
    
    - Supersonic case: disturbance always to the right. If we continuously produce a disturbance, the envelope of the disturbances will define a cone with opening angle $\alpha$ given by
    $$
    \sin \alpha=\frac{c_s}{v} \quad \mathrm{MACH} \mathrm{CONE}
    $$
    \begin{definition}
        {Mach Number}{$M_1= u_1/c_{s,1}$}{This number helps us to simplify the R-H relationships in the next section.}
    \end{definition}
\subsection{The Rankine-Hugoniot Relations}
    We discovered 3 R-H relations, from the conservation of mass, momentum and energy.
    \begin{align*}
        \rho_1 u_1&=\rho_2 u_2\\
        \rho_1 u_1^2+p_1&=\rho_2 u_2^2+p_2\\
        \frac{1}{2} u_1^2+\mathcal{E}_1+\frac{p_1}{\rho_1}&=\frac{1}{2} u_2^2+\mathcal{E}_2+\frac{p_2}{\rho_2}
    \end{align*}
    \subsubsection{What is $\mathcal{E}$}
    For an ideal gas, we have
    $$
    \begin{aligned}
    & \left.\begin{array}{l}
    \mathcal{E}=C_V T \\
    p=\frac{\mathcal{R}_*}{\mu} \rho T
    \end{array}\right\} \Rightarrow \mathcal{E}=\frac{C_V \mu}{\mathcal{R}_*} \frac{p}{\rho} \\
    & \left.\begin{array}{l}
    \gamma=\frac{C_p}{C_V} \\
    C_p-C_V=\frac{\mathcal{R}_*}{\mu}
    \end{array}\right\} \Rightarrow C_V(\gamma-1)=\frac{\mathcal{R}_*}{\mu} \\
    &
    \end{aligned}
    $$
    which combine to give
    $$
    \mathcal{E}=\frac{1}{\gamma-1} \frac{p}{\rho}
    $$
    (internal energy per unit mass)
    \subsubsection{Adiabatic Case}
    The last R-H relationship becomes, in an adiabatic ideal gas:
    \begin{align*}
        \frac{1}{2} u_1^2+\frac{c_{s, 1}^2}{\gamma-1}&=\frac{1}{2} u_2^2+\frac{c_{s, 2}^2}{\gamma-1}
    \end{align*}
    Using all three R-H relations and after some algebra we get
    $$
    \frac{\rho_2}{\rho_1}=\frac{u_1}{u_2}=\frac{(\gamma+1) p_2+(\gamma-1) p_1}{(\gamma+1) p_1+(\gamma-1) p_2}
    $$
    In the limit of strong shocks, $p_2 \gg p_1$, we get
    $$
    \frac{\rho_2}{\rho_1} \rightarrow \frac{\gamma+1}{\gamma-1}
    $$

    \subsubsection{Isothermal Case}
    The last R-H relationship becomes $T_1=T_2$
    with this relationship, we have $c_s^2=u_1u_2$ and 
    $$ 
    \frac{\rho_2}{\rho_1}=\frac{u_1}{u_2}=\left(\frac{u_1}{c_s}\right)^2=M_1^2
    $$
    Note that if $u_1>c_s$ then $u_2<c_s$, flow behind the shock is subsonic.
    \subsection{Supernova Explosion}
\section{Bernoulli's Equation and transonic flows}
\begin{enumerate}
    \item In this section, we started with Bernoulli's equation which gives us Bernoulli Principle, a invaraint term connecting the velocity and $p$, $\rho$, $\psi$.
    \item Then we briefly discuss the vorticities of the fluid.
    \item In the third section, we researched on the De Laval Nozzle, the behaviour of a flow in a tube of area $A$:
    $$
    \left(u^2-c_s^2\right) \nabla \ln u=c_s^2 \nabla \ln A
    $$
    before we applied Bernoulli's principle to the isothermal and adiabatic cases to discuss the physical meaning of the equations.
    \item Lastly, we studied Spherical Accretion and wind, where we considered steady-state and spherically-symmertic accretion flow in the gravitational potential of a central body, i.e.a sun.
\end{enumerate}

\subsection{Bernoulli equation}
This gives us Bernoulli's Principle: For steady barotropic flows, the quantity
$$
H=\frac{1}{2} u^2+\int \frac{\mathrm{d} p}{\rho}+\Psi
$$
is constant along a streamline. The quantity $H$ is called Bernoulli's constant.

If $p=0, H=$ constant is the statement that kinetic $+$ potential energy is constant along streamlines.

If $p \neq 0$, pressure differences accelerate or decelerate the flow as it flows along the streamline.
\subsection{Rotational and irrotational Flow}
Let's start with the momentum equation:
$$
\frac{\partial \mathbf{u}}{\partial t}+(\mathbf{u} \cdot \boldsymbol{\nabla}) \mathbf{u}=-\frac{1}{\rho} \boldsymbol{\nabla} p-\nabla \Psi
$$

\subsection{The De Laval Nozzle}
Laplacian Equation is derived if we have incompressible and irrotational flow. i.e $\div{\vb{u}=0}$ and $\curl{\vb{u}}=0$
we get $$
\begin{aligned}
& u^2 \nabla \ln u=[\nabla \ln u+\boldsymbol{\nabla} \ln A] c_s^2 \\
& \left(u^2-c_s^2\right) \nabla \ln u=c_s^2 \nabla \ln A
\end{aligned}
$$
This implies that an extremum of $A(z)$ must correspond to either
(a) Minimum or maximum in $u$, or
(b) $u=c_s$.
Thus, we see that there is the potential for a transition from subsonic to supersonic flow at a minimum or maximum of the cross-sectional area of the tube.
To make progress, we are applying Bernoulli's equation
$$
\frac{1}{2} u^2+\int \frac{\mathrm{d} p}{\rho}=H, \text { constant } \quad \text { [no gravity, steady, irrotational] }
$$
and examine the two standard barotropic cases.
\subsubsection{Isothermal Case}


\subsection{Spherical Accretion}
In this section, we are considering the symmetric flow of matter in the gravitational potential of a point like a central body, assuming:
\begin{enumerate}
    \item rest at infinity
    \item flow in steady state
    \item barotropic equation of state
\end{enumerate}
By considering mass conservation and momentum conservation for steady states:
$$
\left(u^2-c_s^2\right) \frac{\mathrm{d}}{\mathrm{d} r} \ln u=\frac{2 c_s^2}{r}\left(1-\frac{G M}{2 c_s^2 r}\right)
$$
which means that there is a critical point where:
$$
r=r_s=\frac{G M}{2 c_s^2}
$$
that we call a \textbf{Sonic pint}, where $u$ has an extremum or $u=c_s$\\

\begin{example}
    {Isothermal accretion case, $c_s=\sqrt{\frac{\mathcal{R}_* T}{\mu}}=\mathrm{const}$}
    {We can find the mass accretion rate:
    $$
    \dot{M}=\frac{\pi G^2 M^2 e^{3 / 2} \rho_{\infty}}{c_s^3}
    $$
    }
\end{example}

\begin{example}
    {Polytropic accretion case, $p=K \rho^{1+1 / n}$: }{
        We can find the mass accretion rate:
        $$
        \dot{M}=\frac{\pi(G M)^2 \rho_{\infty}}{c_{s, \infty}^3}\left(\frac{n}{n-\frac{3}{2}}\right)^{n-3 / 2}
        $$
        we can recover the isothermal case by taking the limit:
        $$
        n \rightarrow \infty
        $$
    }
\end{example}
Note that both situations has $\dot{M} \propto M^2$.
\subsubsection{Dependency of the accretion rate}
\begin{enumerate}
    \item Dependence of accretion rate on the mass of an object
    $$
    \dot{M}=A M^2
    $$
    - If initial mass is $M_0$ and it accretes like this for time $t$, then we can integrate to get
    $$
    \int_{M_0}^M \frac{\mathrm{d} M}{M^2}=A t \quad \Rightarrow \quad M=\frac{M_0}{1-A M_0 t}
    $$
    so $\quad M \rightarrow \infty$ as $t \rightarrow 1 / A M_0$
    - In reality, the accretion rate will become limited by fuel supply and/or the Eddington limit (which has $M \dot{\propto} M$, so exponential growth).
    \item Dependence of accretion rate on reservoir properties:
    $$
    \dot{M} \propto \frac{\rho_{\infty}}{c_{\infty}^3} \propto \frac{p_{\infty}}{c_{\infty}^5}
    $$
    - Much higher accretion rates from colder material.
    \item The $n=3 / 2$ is a singular case:
    - Sonic point goes to origin, with infinite sound speed and density
    $$
    \begin{aligned}
    c_s^2 & =\left(\frac{n}{n-\frac{3}{2}}\right) c_{s, \infty}^2, \rho_s=\left(\frac{n}{n-\frac{3}{2}}\right)^n \rho_{\infty} \\
    r_s & =\frac{G M}{2 c_s^2} \rightarrow 0 \text { as } n \rightarrow \frac{3}{2}
    \end{aligned}
    $$
    - But the accretion rate remains finite
    $$
    \dot{M}=\frac{\pi(G M)^2 \rho_{\infty}}{c_{s, \infty}^3}\left(\frac{n}{n-\frac{3}{2}}\right)^{n-3 / 2} \quad \rightarrow \frac{\pi(G M)^2 \rho_{\infty}}{c_{s, \infty}^3} \quad \text { as } n \rightarrow \frac{3}{2}
    $$
    \item Can extend (with less rigor) to the case of the mass moving at speed $v_{\infty}$ through a uniform medium. Accretion rate
    $$
    \dot{M} \sim \frac{(G M)^2 \rho_{\infty}}{\left(c_{\infty}^2+v_{\infty}^2\right)^{3 / 2}} \quad \quad \quad \begin{aligned}
    & \text { BONDI-HOYLE } \\
    & \text { ACCRETION }
    \end{aligned}
    $$
\end{enumerate}
    \subsection{Summary of De Laval Nozzle and Accretion}
    Similarities:

    They all start from the conservation of mass:
    $$
    \rho u A = \dot{M}
    $$
    Take the logarithm of the mass equation:
    $$
    \ln u + \ln \rho + \ln A = \ln \dot{M}
    $$
    Then rewrite the momentum equation in terms of the derivative of log terms.

    In De Laval Nozzle, we need to find A as a function of r, and we can find the maximum velocity at the throat of the nozzle.

    In accretion, we get u as a function of r, and we can find the critical point where $u$ is extremised.

    Then we discuss the isotropic and adiabatic case for De Laval Nozzle and accretion.

\section{Fluid Instabilities}
We say that for a fluid in a steady state, ($\partial{t} = 0$), this is in a state of equilibrium. This results in:
\begin{enumerate}
    \item small perturbation that grows with time and unstable
    \item small perturbation that decays with time or oscillates around the equilibrium configuration, the configuration is stable.
\end{enumerate}
In this section, we explore the example of instabilities, namely \textbf{Convective Instability, Jeans Instability, Rayleigh-Taylor and Kelvin Helmholtz Instability, Thermal Instability}.
\subsection{Convective Instability}
For convective instabilities, we consider an equilibrium system under a uniform field.
Consider a system that has ($p,\rho$) that moves to a surrounding medium of ($p',\rho'$).As its pressure changes to $p'$, its denisity that undergoes \textbf{adiabatic} change, will become $\rho^{*}$ instead of $\rho'$.

Now we have to compare $\rho'$ and $\rho^{*}$ to see if the density difference will grow or decay the instability.

A commonly ysed result for stabiltiy of convection is:
$\frac{\rho}{p \gamma} \frac{\mathrm{d} p}{\mathrm{~d} z}<\frac{\mathrm{d} \rho}{\mathrm{d} z}$

Hence, we have the Schwarzschild stability criterion which reads
$$
\frac{\mathrm{d} T}{\mathrm{~d} z}>\left(1-\frac{1}{\gamma}\right) \frac{T}{p} \frac{\mathrm{d} p}{\mathrm{~d} z}
$$
Since hydrostatic equilibrium requires $\mathrm{d} p / \mathrm{d} z<0$, we see that (since $\gamma>1$ )
- Always stable to convection if $\mathrm{d} T / \mathrm{d} z>0$;
- Otherwise, can tolerate a negative temperature gradient provided
$$
\left|\frac{\mathrm{d} T}{\mathrm{~d} z}\right|<\left(1-\frac{1}{\gamma}\right) \frac{T}{p}\left|\frac{\mathrm{d} p}{\mathrm{~d} z}\right|
$$
So convective instability develops when $T$ declines too steeply with increasing height.
\subsection{Jean Instabilitiy}
In this section, we study the stability of a self-gravitating fluid against gravitational collapse.
\begin{enumerate}
    \item Uniform medium initially static
    \item Barotropic EoS
    \item Gravitational field generated by itself
\end{enumerate}
Applying these condition:
\begin{enumerate}
    \item So equilibrium is
    $$
    \begin{aligned}
    & p=p_0, \text { const. } \\
    & \rho=\rho_0, \text { const. } \\
    & \mathbf{u}=\mathbf{0}
    \end{aligned}
    $$
    \item and governing equations are
    $$
    \begin{aligned}
    & \frac{\partial \rho}{\partial t}+\boldsymbol{\nabla} \cdot(\rho \mathbf{u})=0 \\
    & \frac{\partial \mathbf{u}}{\partial t}+(\mathbf{u} \cdot \boldsymbol{\nabla}) \mathbf{u}=-\frac{1}{\rho} \nabla p-\nabla \Psi \\
    & \nabla^2 \Psi=4 \pi G \rho
    \end{aligned}
    $$
    \item Introduce a perturbation:
    $$
    \begin{aligned}
    & p=p_0+\Delta p \\
    & \rho=\rho_0+\Delta \rho \\
    & \mathbf{u}=\Delta \mathbf{u} \\
    & \Psi=\Psi_0+\Delta \Psi
    \end{aligned}
    $$
    \item Linearized equations are:
    $$
    \begin{aligned}
    & \frac{\partial \Delta \rho}{\partial t}+\rho_0 \nabla \cdot(\Delta \mathbf{u})=0 \\
    & \frac{\partial \Delta \mathbf{u}}{\partial t}=-\frac{\mathrm{d} p}{\mathrm{~d} \rho} \frac{1}{\rho_0} \nabla(\Delta \rho)-\nabla(\Delta \Psi)=-c_s^2 \frac{\nabla(\Delta \rho)}{\rho_0}-\nabla(\Delta \Psi) \\
    & \nabla^2(\Delta \Psi)=4 \pi G \Delta \rho
    \end{aligned}
    $$
    \item Look for plane wave solutions
    $$
    \begin{aligned}
    & \Delta \rho=\rho_1 e^{\mathrm{i}(\mathbf{k} \cdot \mathbf{x}-\omega t)} \\
    & \Delta \Psi=\Psi_1 e^{\mathrm{i}(\mathbf{k} \cdot \mathbf{x}-\omega t)} \\
    & \Delta \mathbf{u}=\mathbf{u}_1 e^{\mathrm{i}(\mathbf{k} \cdot \mathbf{x}-\omega t)}
    \end{aligned}
    $$
    \item Substitution into the linear equations gives: \note{We are solving $\omega$}
    $$
    \begin{aligned}
    & \quad \Rightarrow \quad-\omega \rho_1+\rho_0 \mathbf{k} \cdot \mathbf{u}_1=0 \\
    & \Rightarrow \quad-\rho_0 \omega \mathbf{u}_1=-c_s^2 \mathbf{k} \rho_1-\rho_0 \mathbf{k} \Psi_1 \\
    & \Rightarrow \quad-k^2 \Psi_1=4 \pi G \rho_1 \text {. } \\
    &
    \end{aligned}
    $$
    Eliminating $\mathbf{u}_1$ and $\Psi_1$ from these
    $$
    \begin{aligned}
    \quad \Rightarrow \quad \rho_1 \omega^2 & =k^2\left(\rho_1 c_s^2+\rho_0 \Psi_1\right) \\
    & =k^2 \rho_1 c_s^2-4 \pi G \rho_0 \rho_1 \quad \\
    \Rightarrow \quad \omega^2 & =c_s^2\left(k^2-\frac{4 \pi G \rho_0}{c_s^2}\right) .
    \end{aligned}
    $$
\begin{definition}
    {Jeans wavenumber}{$k_J^2=4 \pi G \rho_0 / c_s^2$}{Determines dispersion realtionship, stability}
\end{definition}
\begin{definition}
    {Jeans Mass}{$M_J \sim \rho_0 \lambda_J^3$}{Mass associated with Jean's wavelength}
\end{definition}
\end{enumerate}


\subsection{Rayleigh-Taylor and Kelvin-Helmholtz Instabilitiy}
In this section, we concern the stability of an interface
\subsection{Thermal Instatbility}
In this section, we consider a system in the absence of a gravitational field and static thermal equilibrium.
Note that $log(K)$ is entropy like 
\begin{enumerate}
    \item Start by rewriting energy equation:
    $$
        \frac{1}{K} \frac{\mathrm{D} K}{\mathrm{D} t}=-(\gamma-1) \frac{\rho \dot{Q}}{p}
    $$
    ENTROPY FORM OF ENERGY EQN
    \item Then combine with coontinuity and momentum equations
    We seek solutions of the form
    $$
    \begin{aligned}
    & \Delta p=p_1 e^{\mathrm{i} \mathbf{k} \cdot \mathbf{x}+q t} \\
    & \Delta \rho=\rho_1 e^{\mathbf{i} \mathbf{k} \cdot \mathbf{x}+q t} \\
    & \Delta \mathbf{u}=\mathbf{u}_1 e^{\mathrm{i} \mathbf{k} \cdot \mathbf{x}+q t} \\
    & \Delta K=K_1 e^{\mathbf{i} \cdot \mathbf{x}+q t}
    \end{aligned}
    $$
    so, instability if $\operatorname{Re}(q)>0$. Substituting into linearized equations gives
    $$
    \begin{aligned}
        &\Rightarrow \quad q \rho_1+\rho_0 \mathbf{i} \mathbf{k} \cdot \mathbf{u}_1=0\\
        &\Rightarrow \quad q \rho_0 \mathbf{u}_1=-\mathrm{i} \mathbf{k} p_1\\
        &\Rightarrow \quad q K_1=-A^* p_1-B^* \rho_1\\
        &\Rightarrow \quad p_1=\rho_0^\gamma K_1+\frac{\gamma p_0}{\rho_0} \rho_1
    \end{aligned}
    $$
    \item We aim to solve $q$, which (whether real or not)determines the stability of the system.
    $$
    q^3+A^* \rho_0^\gamma q^2+k^2 \gamma \frac{p_0}{\rho_0} q-B^* k^2 \rho_0^\gamma=0
    $$
    \item So the system is unstable if $B^*>0$
    \item $\dot{Q}$ is rate cooling - heating. If an increase, $T$ lead to a larger rate of cooling, then stable, otherwise unstable.
    \item Moreover,the Field can be unstable if $A^*<0$ This happens at small $k$
    \item We discuss long wavelength and short wavelength separately.
\end{enumerate} 
\section{Viscosity}
Previously, we considered pressure forces and gravity as the reason for the change in momentum.
This is under the assumption that particles have vanishingly small collisional free paths.
In this section, we consider how momentum is able to diffuse through fluid and this is the concept of viscosity
\subsection{Basic of Viscosity}
The momentum equation needs to be changed
$$
\frac{\partial}{\partial t}\left(\rho u_i\right)=-\partial_j \sigma_{i j}+\rho g_i, \quad g_i=-\partial_i \Psi
$$
with
$$
\sigma_{i j}=\rho u_i u_j+p \delta_{i j}-\underbrace{\sigma_{i j}^{\prime}}_{\begin{array}{c}
\text { viscous } \\
\text { stress tensor }
\end{array}}
$$
As we'll see later, $\sigma_{i j}^{\prime}$ is related to velocity gradients.

Observations about the shear viscosity:
- $\eta$ is independent of density (a denser gas has more particles to transport the momentum but the mean-free-path is shorter);
- $\eta$ increases with $T$
- Isothermal system has $\eta=$ const.
\subsection{Navier-Stroke Equation}
The most general form of $\sigma_{i j}^{\prime}$ which is
- Galilean invariant;
- Linear in velocity components;
- Isotropic
is given by
$$
\sigma_{i j}^{\prime}=\eta\left(\partial_j u_i+\partial_i u_j-\frac{2}{3} \delta_{i j} \partial_k u_k\right)+\zeta \delta_{i j} \partial_k u_k
$$
with $\eta$ and $\zeta$ independent of velocity. This is a symmetric tensor that ensures that there aren't unbalanced torques on fluid elements.
\subsection{Vorticity in Viscous Flows}
Start with the Navier-Stokes equation with $\zeta=0$ and $\eta=$ const.. Take the curl of this, recalling the definition of the vorticity $\mathbf{w}=\nabla \times \mathbf{u}$ :
$$
\frac{\partial \mathbf{w}}{\partial t}=\nabla \times(\mathbf{u} \times \mathbf{w})+\frac{\eta}{\rho} \nabla^2 \mathbf{w}
$$
where, in the last step, we have ignored gradients of $\nu=\eta / \rho$ (so strictly assumed uniform density). So, vorticity is carried with the flow but also diffuses through flow due to the action of vorticity.
\subsection{Energy dissipation}
Viscosity leads to the dissipation of kinetic energy into heat — an irreversible process.
Let's analyze this in the case of an incompressible flow so that we don't need to about $p d V$ work. Then the total kinetic energy is
$$
E_{\text {kin }}=\frac{1}{2} \int \rho u^2 \mathrm{~d} V
$$
Let's consider the rate of change of $E_{\text {kin }}$ with time, using the momentum euqation
$$
\begin{aligned}
\frac{\partial}{\partial t}\left(\frac{1}{2} \rho u^2\right) & =u_i \frac{\partial}{\partial t}\left(\rho u_i\right) \\
& =-u_i \partial_j\left(\rho u_i u_j\right)-u_i \partial_j \delta_{i j} p+u_i \partial_j \sigma_{i j}^{\prime} \\
& =-u_i \partial_j\left(\rho u_i u_j\right)-u_i \partial_i p+\partial_j\left(u_i \sigma_{i j}^{\prime}\right)-\sigma_{i j}^{\prime} \partial_j u_i
\end{aligned}
$$
\subsection{Viscous Flow through a Pipe}
Now consider flowing through a long pipe with a constant circular cross-section.
Navier-Stokes equation reads
$$
\begin{gathered}
\underbrace{\frac{\partial \mathbf{u}}{\partial t}}_{\begin{array}{c}
\text { steady } \\
\text { state = 0}
\end{array}}+\underbrace{\mathbf{u}\cdot\nabla \overrightarrow{\mathbf{u}}}_{\text {symmetry = 0}} =-\frac{1}{\rho} \nabla p+\nu[\nabla^2 \mathbf{u}+\underbrace{\frac{1}{3} \nabla(\nabla \cdot \mathbf{u})}_{\text {incompressible = 0}}] \\
\Rightarrow \quad \nu \nabla^2 \mathbf{u}=\frac{1}{\rho} \nabla p
\end{gathered}
$$
We only need to consider the z-component of the equation:
Integrating gives
$$
u=-\frac{\Delta p}{4 \rho \nu l} R^2+a \ln R+b
$$
$$
\Rightarrow \quad u=\frac{\Delta p}{4 \nu \rho l}\left(R_0^2-R^2\right)
$$
So velocity profile is parabolic.
The mass flux passing through an annular element $2 \pi R \mathrm{~d} R$ is $2 \pi R \rho u \mathrm{~d} R$. So, the total mass flow rate is
$$
Q=\int_0^{R_0} 2 \pi \rho u R \mathrm{~d} R=\frac{\pi}{8} \frac{\Delta p}{\nu l} R_0^4
$$
As $\eta \rightarrow 0$, i.e. $\nu \rightarrow 0$, the flow rate $\rightarrow \infty$ (or, in other words, an inviscid flow can't be in steady state in this pipe if there is a non-zero pressure gradient).
If $\Delta p$ increases sufficiently, it becomes unstable and irregular, giving turbulent motions above critical speed.

\note{ In SCM, this is the same as the Poiseuille flow in a channel of circular cross-section}
\subsection{ Accretion Disk}
There are two ways that angular moemntum is lost:
\begin{enumerate}
    \item Wind: Fluid elements in a disk pass their angular momentum vis magnetic forces to material that carries its way in the wind.
    \item Internal Viscosity: Viscosity will allow angular momentum to be transferred from the fast-moving inner regions to the more slowly-moving outer regions. This means the inner disk fluid elements lose angular momentum.
\end{enumerate}
Consider the change in mass:
$$R \frac{\partial \Sigma}{\partial t}=-\frac{\partial}{\partial R}\left(R \Sigma u_R\right)$$
Considering the conservation of moementum:
$\frac{\partial}{\partial t}\left(R \Sigma u_\phi\right)=-\frac{1}{R} \frac{\partial}{\partial R}\left(\Sigma R^2 u_\phi u_R\right)+\frac{1}{R} \frac{\partial}{\partial R}\left(\nu \Sigma R^3 \frac{\mathrm{d} \Omega}{\mathrm{d} R}\right)$
Combines continuity equation and conservation of momentum and specialize to the case of a Newtonian point-source gravitational field $\Omega=\sqrt{G M / R^3}$ gives
$$
\frac{\partial \Sigma}{\partial t}=\frac{3}{R} \frac{\partial}{\partial R}\left[R^{1 / 2} \frac{\partial}{\partial R}\left(\nu \Sigma R^{1 / 2}\right)\right]
$$
So the surface density $\Sigma(R, t)$ obeys a diffusion equation.

\section{Plasmas}
This section discusses the particles in magnetic fields.
\end{document}