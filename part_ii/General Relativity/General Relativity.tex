\documentclass[12pt,a4paper]{article}

\usepackage{import}
\import{../Template/}{format.tex}

\newcommand{\topic}{General Relativity}

\begin{document}

\begin{titlepage}
    \maketitle
\end{titlepage}

\tableofcontents

\newpage

\begin{abstract}
\noindent
Abstract of this course
\end{abstract}

\section{Particle Dynamics}
\subsection{Lorentz transformation}
\begin{definition}
    {homogeneous Lorentz transformation}
    {$x^{'\mu} = {\Lambda ^ \mu}_\nu x^\nu$}
    {...}
\end{definition}
\begin{definition}
    {transformation matrix}
    {${\Lambda^\mu}_\nu = \mqty(\dmat{\gamma & -\beta\gamma \\ -\beta\gamma & \gamma,1,1})$}
    {This is the end }
\end{definition}


\section{Electromagnetism}
\begin{definition}
    {Electromagnetic field tensor}
    {$F\mu\nu = \partial_\mu A_\nu - \partial_\nu A_\mu$}
    {$F\mu\nu$ is antisymmetric by construction and contains four independents fields }
\end{definition}

\section{Spacetime Curvature}
\definition{Riemann Curvature Tensor}
    {$R_{abc}^d=\nabla_a\nabla_b v_c-\nabla_b\nabla_a v_c$}
    {The reason for doing so is that covariant derivative of dual vector field is not commutative}
    \subsection{Subsection 1}
\section{Gravitation field equations}
    \subsection{Energy Momentum Tensor}
    \definition{Energy Momentum Tensor}
        {$T^{\mu\nu}(x)=\rho_0(x) u^\mu(x) u^\nu(x)$}
        {
        $T^{00}$: energy density\\
        $T^{i0}$: i-th component of 3-momentum density (times c)\\
        $T^{ij}$: flux of i-component of 3-momentum in j-direction
        }
    \subsubsection{Properties of Energy-momentum tensor}
    \begin{itemize}
        \item Always symmetric
    \end{itemize}
    \subsection{Energy-momentum Tensor in Ideal Fluid}
        Consider Ideal Fluid:\\
        We can find a local inertial frame so that $T^{i0}=0$; and the spatial components are isotropic: $T^{ij}\propto\delta^{ij}$.\\
        Hence, in \textbf{instantaneous rest frame}:
        \begin{align*}
            T^{\mu\nu}= \text{diag}(\rho c^2,p,p,p)
        \end{align*}
        where $\rho c^2$ is the rest frame energy density and $p$ is isotropic pressure.
        While in general:
        \begin{equation}\label{Energy-momentum Tensor}
            T^{\mu\nu}= (\rho+\dfrac{p}{c^2})u^\mu u^\nu - pg^{\mu\nu}
        \end{equation}
        Equation \ref{Energy-momentum Tensor} is the energy momentum tensor, valid in any coordinate system.\\
        \subsubsection{Continuity Equation}
        The energy-momentum tensor satisfies continuity equation:
        \begin{equation*}
            \nabla_\mu T^{\mu\nu} = 0
        \end{equation*}
    \subsection{Einstein Field Equation}
    \begin{definition}
        {Einstein Field Equation}
        {$G_{\mu\nu}=R_{\mu\nu}-\dfrac{1}{2}g_{\mu\nu}R=-\kappa T_{\mu\nu}$}
        {Constant proportionality and negative sign on the right is required for consistency with the weak field limit}
    \end{definition}
\section{The Schwarzchild Solution}
Adopting a passive view point: change the coordinate system without changing the functional form of the fields on our coordinates.
\subsection{Geodesics in Schwarzchild spacetime}
In this section we study the equation of motion for 4 coordinates, $t,r,\theta,\phi$
    \begin{enumerate}
        \item Equation of motion for $\theta$:
        \begin{equation}
            \ddot{\theta}+\frac{2}{r}\dot{r}\dot{\theta} - \sin{\theta}\cos{\theta}\dot{\psi}^2 = 0
        \end{equation}
        A possible solution is $\theta = \pi/2$, planar motion in the equatorial plane; given the spherical symmetry.
        \item Equation of motion for $t$:
        \begin{equation}
            (1-\frac{2\mu}{r})\dot{t} = k
        \end{equation}
        $k = (1-2\mu/r)\dot {t}$ is related to the energy of the particle as measured by stationary observer.
        \item Equation of motion for $\phi$:
        \begin{equation}
            r^2\dot{\phi} = h
        \end{equation}
        Here, we assume in the plane $\theta= \pi/2$. $h$ arises from the symmetry of the spacetime under ratotion about z-axis, can be interpreted as $specific \ angular\ momentum$.
        \item Equation of motion for $r$:\\
        \begin{equation}
            (1-\frac{2\mu}{r})c^2\dot{t}^2 - (1-\frac{2\mu}{r})^{-1} \dot{r}^2 - r^2\dot{\phi}^2 =
            \begin{cases}
                c^2 & \text{massive}\\
                0 & \text{massless}
            \end{cases}
        \end{equation}
    \end{enumerate}
    \subsection{Effective potential energy}
        \begin{enumerate}
            \item In spherical coordinates, the Newtonian effective potential energy is 
            \begin{equation}
                V_{\text{eff}}(r) = -\frac{GM}{r}+\frac{h^2}{2r^2}
            \end{equation}
            \begin{itemize}
                \item It has a centrifugal barrier at small r, preventing particles reaching $r=0 $
                \item Bound orbits have $E_N < 0$, two turning points for $r$ w.r.t $V$ at $V=E_N$
                \item Effective potential have one turning point at $r=h^2/GM$ It is a minimum, corresponding to stable circular orbit.
            \end{itemize}
            \item massive  particle in general relativity:
            \begin{equation}
                V_{\text{eff}}(r) = -\frac{GM}{r}+\frac{h^2}{2r^2}(1-\frac{2\mu}{r})
            \end{equation}
            \begin{itemize}
                \item the centrifugal term is slightly modified by a factor of $(1-2\mu/r)$
                \item solving for the extrema of V gives:
                \begin{equation}
                    r_{\pm}=\frac{h}{2\mu c^2}(h\pm\sqrt{h^2-12\mu^2 c^2})
                \end{equation}
                \item two stationary points for $h>\sqrt{\mu c}$, $r_{-}$ is a maximum and $r_{+}$is minimum
                \item none for smaller h i.e V is increasing with r
                \item 
            \end{itemize}

        \end{enumerate}
        

\section{Shapes of orbits for massive and massless particles}
\subsection{Shape of Orbit}
    \begin{enumerate}
        \item Massive article gives
        \begin{equation}
            \frac{d^2 u}{d\phi^2} + u -3 \mu u^2 = \frac{GM}{h^2}
        \end{equation}
        \item Massless article gives
        \begin{equation}
            \frac{d^2 u}{d\phi^2} + u -3 \mu u^2 = 0
        \end{equation}
    \end{enumerate}


\end{document}