\documentclass[12pt,a4paper]{article}
\author{Xinyu Zhong\\Wolfson College}
\usepackage{physics, amsmath}
\usepackage{xcolor}
\usepackage[margin=0.5in]{geometry}

\title{Notes}
%\date{2nd Nov 2021}

\usepackage{fancyhdr}
\pagestyle{fancyplain}
\fancyhf{}
\lhead{\fancyplain{}{Zachary Zhong, xz447@cam.ac.uk}}
\rhead{\fancyplain{}{Thermodynamics}}
\cfoot{\fancyplain{}{\thepage{}}}
\setlength {\headheight}{15pt}

\newcommand{\definition}[3]
    {
    \textit{Definition #1: }
    \begin{center}
        {#2}
    \end{center}
    {#3}
    }
\newcommand{\theorem}[2]{\textbf{\textcolor{red}{#1: }}\textcolor{red}{#2}}
\newcommand{\example}[1]{\par\textbf{Example: }\textcolor{blue}{#1}}

\begin{document}

\begin{titlepage}
    \maketitle
\end{titlepage}

\tableofcontents

\newpage

\begin{abstract}
\noindent
Abstract of this course
\end{abstract}


\section{Spacetime Curvature}
\definition{Riemann Curvature Tensor}
    {$R_{abc}^d=\nabla_a\nabla_b v_c-\nabla_b\nabla_a v_c$}
    {The reason for doing so is that covariant derivative of dual vector field is not commutative}
    \subsection{Subsection 1}
\section{Gravitation field equations}
    \subsection{Energy Momentum Tensor}
    \definition{Energy Momentum Tensor}
        {$T^{\mu\nu}(x)=\rho_0(x) u^\mu(x) u^\nu(x)$}
        {
        $T^{00}$: energy density\\
        $T^{i0}$: i-th component of 3-momentum density (times c)\\
        $T^{ij}$: flux of i-component of 3-momentum in j-direction
        }
    \subsubsection{Properties of Energy-momentum tensor}
    \begin{itemize}
        \item Always symmetric
    \end{itemize}
    \subsection{Energy-momentum Tensor in Ideal Fluid}
        Consider Ideal Fluid:\\
        We can find a local inertial frame so that $T^{i0}=0$; and the spatial components are isotropic: $T^{ij}\propto\delta^{ij}$.\\
        Hence, in \textbf{instantaneous rest frame}:
        \begin{align*}
            T^{\mu\nu}= \text{diag}(\rho c^2,p,p,p)
        \end{align*}
        where $\rho c^2$ is the rest frame energy density and $p$ is isotropic pressure.
        While in general:
        \begin{equation}\label{Energy-momentum Tensor}
            T^{\mu\nu}= (\rho+\dfrac{p}{c^2})u^\mu u^\nu - pg^{\mu\nu}
        \end{equation}
        Equation \ref{Energy-momentum Tensor} is the energy momentum tensor, valid in any coordinate system.\\
        \subsubsection{Continuity Equation}
        The energy-momentum tensor satisfies continuity equation:
        \begin{equation*}
            \nabla_\mu T^{\mu\nu} = 0
        \end{equation*}
    \subsection{Einstein Field Equation}
    \begin{definition}
        {Einstein Field Equation}
        {$G_{\mu\nu}=R_{\mu\nu}-\dfrac{1}{2}g_{\mu\nu}R=-\kappa T_{\mu\nu}$}
        {Constant proportionality and negative sign on the right is required for consistency with the weak field limit}
    \end{definition}
\section{The Schwarzchild Solution}
Adopting a passive view point: change the coordinate system without changing the functional form of the fields on our coordinates.



\end{document}