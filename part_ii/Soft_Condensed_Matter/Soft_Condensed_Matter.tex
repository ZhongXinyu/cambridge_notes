\documentclass[12pt,a4paper]{article}

\usepackage{import}
\import{../Template/}{format.tex}

\newcommand{\topic}{Soft Condensed Matter}

\begin{document}

\title{\topic}
\begin{titlepage}
    \maketitle
\end{titlepage}

\tableofcontents

\newpage
\begin{abstract}
\noindent
Abstract of this course
\end{abstract}

\section{Element of fluid dynamics}
This section devotes itself to the basics of fluid dynamics,
    \subsection{Fluid Dynamics equations}
        \subsubsection{Mass flux and continuity equation}
            \begin{enumerate}
                \item  Rate of change in mass in volume integral of change in density.
                \item  Conservation of mass states that the rate of change is equal to the net flux of mass flowing, 
                followed by a divergence theorem to change surface integral to volume integral.
                \item Combining the rate of change in mass and conservation of mass, as both are volume integral:            
                \begin{equation}
                    \pdv{\rho}{t} = -\div(\rho \vb{v})
                \end{equation}
                \item For incompressible fluid, $\rho$ is constant, and the continuity equation simplify to $\div\vb{v} = 0$.
            \end{enumerate}
        \subsubsection{Momentum Flux and the equation of motion}
            \begin{enumerate}
                \item Rate of change in total momentum $P_i(V,t)$
                \begin{equation}
                    \pdv{P_i(V,t)}{t}=\pdv{t} \int_{V}d^3r\rho v_i=\int_{V}(\pdv{t}\rho) v_i+\rho\pdv{t}v_i
                \end{equation}
                \item The change in momentum is the sum of the forces, which take into account 4 contributions, \textbf{Convention of momentum, Pressure Forces, Viscous Forces, Body Forces}.
                \item Overall stress tensor 
                \begin{equation}
                    \sigma_{ij}=\sigma' {ij} - p\delta{ij}
                \end{equation}
                where the two items are contribution for viscous pressure and pressure forces
                \item 
            \end{enumerate}
        \subsection{Navier-Stokes equation}
            \begin{enumerate}
                \item The complete Navier-Stokes equation
                \begin{equation}
                    \rho [\partial_t\vb*{v} + (\vb*{v}\dotproduct \nabla)\vb*{v}]= - \grad{p} +\eta\laplacian{v}+\rho\vb*{g}
                \end{equation}
                \item Left-hand side relate to inertial force densities
                \item Right-hand side encompass intrinsic viscosity and applied force densities (pressure gradient and gravity)
                \item Conservative external forces, including gravity, the eternal force term is absorbed into pressure term.
            \end{enumerate}
        \subsection{Material Derivative}
            \begin{enumerate}
                \item Concept of streamline
                \item Lagrangian rate of 
                \begin{equation}
                    \frac{Df}{Dt} = \pdv{f}{t} = (\vb*{v} \dotproduct \vb{\nabla})f
                \end{equation}
                \item In particular, the acceleration of fluid element is:
                \begin{equation}
                    \frac{D\vb{v}}{Dt} = \pdv{\vb{v}}{t} = (\vb*{v} \dotproduct \vb{\nabla})\vb{v}
                \end{equation}
                \item 
            \end{enumerate}
        \subsection{Reynold Number and Stokes Flow}
            \begin{enumerate}
                \item Limit of low flow velocities, or small sizes, 
                \item Reynold number 
                \begin{equation}
                    Re =\frac{\rho v_0 L_0}{\eta}
                \end{equation}
                \item Low Reynolds number $Re << 1$, the viscous term dominates over inertia, non-linear Navier-Stokes equation is reduced to linear Stokes equation:
                \begin{equation}
                    0 = -\grad p +\eta\laplacian \vb{v}
                \end{equation}
                \item If an external force is applied, for instance an oscillating boundary, time derivative is given by the external force and is not negligible, which gives time-dependent Navier Stokes Equation
                \begin{equation}
                    \rho \partial_{v}\vb*{v} = -\grad p +\eta\laplacian \vb*{v}
                \end{equation}
            \end{enumerate}
        \subsection{Properties of Stokes flow and locomotion of microorganism and nanomachines}
            \begin{enumerate}
                \item Since time doesn't enter explicitly the Stokes equation, and it is linear, the flow pattern is unchanged when the pressure is increased, only the flow velocity is changed.
                \item If reverse spacial direction, the pressure gradient changes sign while the Laplacian keeps its sign, the flow velocity changes sign, therefore Stoke equation is unchanged under reversal of spatial coordinates.
                \item Kinetic reversibility properties of Stokes flow implies that a microorganism attempts
                \item This observation has interesting consequences for locomotion on small scales as is applicable to microorganisms or artificial nanomachines. 
                Indeed, the kinematic reversibility property of Stokes flows implies that a microorganism which attempts to swim through a reversible sequence of changes of shape, returning to its original shape by going through the sequence in reverse, will not translate, since any motion that it undergoes in the first half of the cycle will be reversed in the second part of the swimming cycle. 
                This is known as the Scallop theorem (originally by Edward Purcell). 
                A scallop swims to a good approximation by opening and closing a single hinge. 
                At low Reynolds numbers, the forward movement upon opening would be exactly cancelled by the motion during the reverse stroke and thus the scallop would remain stationary. 
                A real scallop is able to avoid this problem by closing the hinge very rapidly, escaping the low Reynolds number regime. 
                This is made possible by its size such an escape becomes increasingly difficult for smaller scale objects such as microorganisms and artificial nanorobots.
                \item Different strategy to break time-reversal symmetry to operator with circular motion using rotary molecular motors. Example like E.coli
            \end{enumerate}
        \subsection{Vorticity}
        \subsection{Fluid in mechanical equilibrium}
            \begin{enumerate}
                \item Viscous Fluid in mechanical equilibrium has to be at rest, $\vb*{v}=0$ relative to contain.
                \item Navier-Stokes equation reduce to:
                \begin{equation}
                    0 = -\grad{p} -\rho g \vb*{e}_z
                \end{equation}
                \item Incompressible fluid, integrate this equation to yield:
                \begin{equation}
                    p(z) = p^{*} - \rho g z
                \end{equation}
                where $p^{*}$ is constant of integration, which represents the pressure at z = 0 and the z dependent constribution to the pressure is given as the hydrostatic pressure:
                \begin{equation}
                    p_{hs} = - rho gz
                \end{equation}
                We can therefore regroup the effects of gravity into the pressure term and write the total pressure as:
                \begin{equation}
                    p_{tot} = p + p{hs}
                \end{equation}
                \item 
            \end{enumerate}
        \subsection{Couette Flow}
            Couette flow is  a flow induced in a liquid through the movement of one or more walls. No pressure gradiant.  
            \begin{example}
                {Planar geometry}{
                    Bottom plate at $z=0$ and top plate $z=h$, liquid is moving with speed $v_0$ while stationary at bottom.
                    Use stokes equations, together with boundary condition
                    \begin{equation}
                        \eta \partial^2_z v_x(z) = 0
                    \end{equation}
                    Use viscous stress tensor $\sigma '$ to determine the horizontal force where $F_z=\sigma '_{xz}A = \eta \frac{v_0 A}{h}$
                }
            \end{example}
        \subsection{Oscillatory flow}
            \begin{enumerate}
                \item The equation for motion for the flow field is given by time-dependent Stokes equation
                
                \item Wave solution:
                \begin{equation}
                    v_x = v_0 e^{i\omega t}e^{(-1+i)z/\sigma}
                \end{equation}
                where the parameter $\sigma$ gives the decay length of the wave in the fluid.
                \item Force per area is given by:
                \begin{equation}
                    FA = \eta \partial_z v_x (z=0) = (-1+i) \frac {\eta v}{\sigma}
                \end{equation}
            \end{enumerate}
        \subsection{Diffusion of momentum}
            \begin{enumerate}
                \item Equation of flow field is a diffusion equation for the transverse flow with a diffusion coefficient equal to $\eta/\rho=\nu$, the kinetic viscosity.
                \item Transverse momentum diffuse in the z-direction away from its source at the moving plate.
            \end{enumerate}    
        \subsection{Stokes drag force}
            Consider A spherical body of radius $a$ moving with a velocity $\vb{v}$ through a fluid with is overall stationary. It creates a temporary disturbance in the flow field which disappears after body passes a fixed observation point. The drag force that teh shpere experieences due to the motion of the underlying fluid is $\vb{F} = 6\pi \eta a \vb*{v}$
            \subsubsection{Scaling argument}
            In the low Reynolds number regime, assuming stationary flow and neglecting the pressure gradient term, we have Navier Strokes equation:
            \begin{equation}
                \laplacian{v_T}=0 
            \end{equation}
            \subsubsection{Full analysis}
            **No examinable
        \subsection{Hydrodynamic interaction between colloidal particles}
            As a consequence of the moving colloid. The flow velocity at a distance r decays as 1/r. Through this velocity field, one colloid can exert drag force on another. Such hydrodynamic interaction is \textbf{long ranged and decay slowly as the inverse of the particle separation}.
            This interaction can be expressed by extending Stokes's relation to multiple particle:
            \begin{equation}
                \vb{v} = H \vb{F}
            \end{equation}
            Here H is the mobility matrix
        \subsection{Poiseuille flow} 
            \textbf{Pressure driven} steady states flow through small channels. (Recall that both Couette flow consider system in absence of pressure)
            \begin{example}
                {Parallel Plate Channel}{Fluid flows in x-direction, symmetry in both y and z direction:
                \begin{align*}
                    \partial_z^2 v_x(z) &= -\frac{\Delta p}{\eta L}\\
                    \intertext{Given B.C. At $z=0$ and $h$}
                    v_x(z=0) &=0\\
                    v_x(z=h) &=0\\
                    \intertext{Solution is a parabolic}
                    v_x(z) &= \frac{\Delta p}{2 \eta L}(h-z)z\\
                    \intertext{Overall flow is characterised by the volumetric flow rate, Q}
                    Q &= \int_c dydzv_x(y,z)
                    \intertext{The flow rate through a section of width x, meaning}
                    Q &= \int^\omega_0 dy \int^h_0 dz\frac{\Delta p}{2\eta L}(h-z)z = \frac{h^3\omega}{12\eta L}\Delta p
                \end{align*}
                }
            \end{example}
            \begin{example}
                {Channel with circular cross-section}{
                    \begin{align*}
                        (x,y,z)&=(x, r\cos (\psi), r\sin(\psi))\\
                        \vectorarrow{e}_x &= \vectorarrow{e}_x\\
                        \vectorarrow{e}_r &= \cos \psi \vectorarrow{e}_y + \sin \psi \vectorarrow{e}_z\\
                        \vectorarrow{e}_{\psi} &= -\sin \psi \vectorarrow{e}_y + \cos \psi \vectorarrow{e}_z\\
                        \intertext{Volumetric flow rate:}
                        Q &= \int^{2\pi}_0 d\psi \int^a_0 dr r \frac{\Delta p}{4\eta L}(a^2-r^2) = \frac{\pi a^4}{8\eta L}\Delta p\\
                    \end{align*}
                }
            \end{example}
        \subsubsection{Hydraulic resistance and Compliance}
            \begin{definition}
                {Hydraulic resistance}{$\Delta p = R_{hydr}Q$}
                {
                    The flow rate, $Q$ in a straight channel at a steady state is proportionalto the pressure difference at its ends $\Delta p$. This relationship holds for any channel geomertry and is known as the Hagen-Poiseuille law.
                }
            \end{definition}
            Compare Fluid to Circuits,\\
            \begin{tabular}{ p{5cm}p{5cm}p{7cm}  }
                \hline
                \multicolumn{3}{c}{Summay} \\
                \hline
                                & Ohm's Law&Hagen-Poiseuille's Law\\
                Current & Electric current $I$  & Volumetric flow rate$Q$\\
                Transported quantity    & Charge $q$    & Fluid volume $\nu$\\
                Driving force & Potential Different $\Delta V$ & Pressure  difference $\Delta p$\\
                Resistance& Electric resistance     &Hydraulic resistance $R_{hydr}=\frac{\Delta p}{Q}$\\
                Capacitance & $C=\frac{dq}{dv}$ & Hydraulic Capacitance $C_{hydr}=\frac{d\nu}{dp}$\\
                \hline
               \end{tabular}
        

            
            
\section{Viscoelasticity}
\subsection{Basic Concept of Elasticity}
    \subsubsection{Strain}
        \begin{enumerate}
            \item Definition of strain tensor
            \begin{equation}
                \epsilon_{ik}= \dfrac{1}{2}(\partial_{k}u_i-\partial_{i}u_k)
            \end{equation}
            Strain tensor is symmetric and diagonaisable
        \end{enumerate}
    \subsubsection{Stress}
        \begin{enumerate}
            \item Stress in defined as force $d\vb{f}$ per unit area $dS$ transmitted across the surface element $dS$:
            \begin{equation}
                dF_i=\sigma_{ik} dS_k
            \end{equation}
            \item For example, $\sigma_{xy}$ is the force per unit area in teh x direction transmitted across the plane into the normal vector in the y-direction.
        \end{enumerate}
\subsection{Linear elasticity and Hooke's Law}
    \subsubsection{Hooke's Law}
        \begin{enumerate}
            \item For small strains, there is a linear relationship between stress and strain as stated in the Hookes's law. 
            \begin{equation}
                \sigma_{ij}=C_{ijkl}\epsilon_{kl}
            \end{equation}
            where $C_{ijkl}$ is the stiffness tensor.
            \item Stiffness tensor has element, given that it is symmetric, it has 21 independent elements, of which 3 are related to the orientation of body in space. Therefore, 18 independent tensor element need to be considered.
            \item Consider an \textbf{isotropic linear} elastic medium. For such material, all directions be have the same way and no specific internal directions; as such the \textcolor{red}{The right handside of the equation can onlu ahve elements proportional to the strain tensor itself, or the only scalar combination$\sum_{k}\epsilon_{kk}$ of the matrix element of the strain tensor, as any other combination of matrix elements of the strain tensor would not result in behavior which is identical in all spatial directions}
        \end{enumerate}
    \subsubsection{Poisson's ratio} 
        \begin{enumerate}
            \item When a stress $\sigma_{xx}$ is applied for instance , along x axis, it streches along this direction by strain $\epsilon_{xx}$. Stress and stains are linearly proportional:
            \begin{equation}
                \sigma_{xx} = E \epsilon_{xx}
            \end{equation}
            where $E$ is the Young's modulus, which has unit of \textbf{pressure}.
            \item Material also contract in the other directions, $\epsilon_{yy} = \epsilon_{zz} < 0$. We define Poisson's ration as:
            \begin{equation}
                \nu = -\dfrac{\epsilon_{yy}}{\epsilon_{zz}}
            \end{equation}
            \item The resultant strain tensor is:
            $\begin{pmatrix}
                \epsilon_{xx} & 0 & 0\\
                0 & -\nu \epsilon_{xx} & 0\\
                0 & 0 & -\nu \epsilon_{xx}
            \end{pmatrix}$
            \item and strains due to $\sigma_{xx}$ is given as:
                \begin{align}
                    E\epsilon_{xx}  &= \sigma_{xx}\\
                    E\epsilon_{yy}  &= -\nu \sigma_{xx}\\
                    E\epsilon_{yy}  &= -\nu \sigma_{xx}
                \end{align}
            \item Moreover, the linear system can be solved for $\sigma_{ii}$:
            \begin{equation}
                \sigma_{xx}= \frac{E}{(1+\nu)(1-2\nu)}[(1-\nu)\epsilon_{xx}+\nu \epsilon_{yy}+ \nu\epsilon_{zz}]
            \end{equation}
        \end{enumerate}
    \subsubsection{Bulk Modulus}
        \begin{enumerate}
            \item Bulk modulus measures the contraction of body under isotropic pressure:
            \begin{equation}
                -p= \sigma_{xx}+\sigma_{yy}+\sigma_{zz}
            \end{equation}
            \item Bulk Modulus is defined as:
            \begin{equation}
                p=-B\frac{\Delta V}{V} 
            \end{equation}
            where $\frac{\Delta V}{V} = \Tr{\epsilon}=-3p(1-2\nu)/E$
            \item we get:
            \begin{equation}
                B = \frac{E}{3(1-2\nu)}
            \end{equation}
        \end{enumerate}
    \subsubsection{Shear deformation}
    \subsubsection{Physical Constrains}
        \begin{enumerate}
            \item Bulk Modulus cannot be negative, otherwise would expand
            \item This also means that $\nu <1/2$.
            \item For a perfectly incompressible material, $\nu = 0$
            \item While most materials have $\nu > 0$, there are materials that have negative $\nu$
            \item Shear Modulus $G$ and Young's Modulus $E$ are positive 
        \end{enumerate}
\subsection{Viscoelasticity}
    \paragraph*{Newtonian liquids}
        Do not have shear modulus but resist flow through their viscosity, which is assumed to be constant and independent of the flow rate.
    \paragraph*{Hookean Solids}
        Respond immediately to a stress and change their shape and do not dissipate energy
\subsection{Linear viscoelastic materials and experiments}
    Linear means that a system whose response is linearly proportional to the applied stresses. \textcolor{red}{shear modulus independent of strain.}
    \paragraph*{Creep experiment}
    Stress $\sigma_0$ is applied, response strain $\epsilon$ is measured.
    \paragraph*{Stress relaxation experiment}
    Rapid strain $\epsilon_0$ is applied and maintained, and the decay of stress $\sigma(t)$ is measured.Decay is governed by relaxation modulus $G(t)$, with $\sigma(t) = G(t)\epsilon_0$
\subsection {Time translation invariance}
    In this section, we add incremental stresses accumulated as a function of history of material and see how is strain $\sigma$ behaves.
    \begin{enumerate}
        \item Each of these strain increment will trigger time-dependent increment in stress $d\sigma(t) = G(t-t')d\epsilon(t')$
        Overall stress is linear superposition of all the contribution of $d\sigma(t)$:
        \begin{equation}
            \sigma(t)=\int d\sigma(t) = \int G(t-t')d\epsilon(t') =\int^{t}_{-\infty } G(t-t')\dv{\epsilon(t')}{t'} dt' 
        \end{equation}
        This is a `retarded' linear response.
        \item In the case of complex viscoelastic fluids, we have constant shear rate $\dot{\epsilon}$, therefore:
        \begin{equation}
            \sigma(t) = \dot{\epsilon}\int^{t}_{-\infty}G(t-t')dt'= \int^{\infty}_{0}G(\tau)d\tau
        \end{equation}
        From here we get viscosity
        \begin{equation}
            \eta =\int^{\infty}_{0}G(\tau)d\tau
        \end{equation}
    \end{enumerate}
\subsection{Complex modulus}
    In this section ,we consider response of a stress in response to an applied oscillating strain:
    \begin{equation}
        \epsilon(t)=\epsilon_0 e^{i\omega t}
    \end{equation}
    \begin{enumerate}
        \item Substitute this into previous expression for $\sigma$.
        \item We have:
        \begin{equation}
            \sigma(t) = G^{*}(\omega)\epsilon_0 e^{i\omega t}
        \end{equation}
        where
        \begin{equation}
            G^{*}(\omega) = i\omega \int^{\infty}_{0} G(\tau)e^{-i\omega \tau} d\tau
        \end{equation}
        \item Ideal solid $G(\tau)=G_0$ and $G^{*}(\omega) = G_0$
        \item Newtonian fluid $G(\tau)=\delta(\tau)\eta$ and $G^{*}(\omega) = i \omega \eta$
        \item General visco-elastic material, we have $G^{*}(\omega) = G'+i G''$, where $G'$ describes in-phase response from elastic contribution and $G''$  is the out of phase response from the viscous dissipative contribution.
    \end{enumerate}
\subsection{Simple model of viscoelasic material}
In this section, we discuss some models of viscoelastic material such as Maxwell fluid. 
    \subsubsection{Maxwell fluid}
    Maxwell fluid has a relaxation time scale $\tau$:
    \begin{equation}
        G(t) = G_0 e^{-t/\tau}
    \end{equation}
    \subsubsection{Kelvin-Voigt solid}
    \subsubsection{Zener standard linear solid}
\section{Polymers}
\section{Molecular self-assembly}

\end{document}
