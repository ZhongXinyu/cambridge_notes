\documentclass[12pt,a4paper]{article}
\author{Xinyu Zhong\\Wolfson College}
\usepackage{physics, amsmath}
\usepackage{xcolor}
\usepackage[margin=0.5in]{geometry}
\usepackage{fancyhdr}
\pagestyle{fancyplain}
\fancyhf{}

%\date{2nd Nov 2021}


%\date{2nd Nov 2021}
\lhead{\fancyplain{}{Zachary Zhong, xz447@cam.ac.uk}}
\rhead{\fancyplain{}{Classical Field Theory}}
\cfoot{\fancyplain{}{\thepage{}} }
\setlength{\headheight}{40pt}

\newcommand{\definition}[3]
    {
    \textit{Definition #1: }
    \begin{center}
        {#2}
    \end{center}
    {#3}\\
    }
\newcommand{\theorem}[2]{\textbf{\textcolor{red}{#1: }}{#2}\\}
\newcommand{\example}[2]{\textbf{Example: #1}\\\textcolor{blue}{#2}\\}

\title{Notes}

\begin{document}

\begin{titlepage}
    \maketitle
\end{titlepage}

\tableofcontents

\newpage

\begin{abstract}
\noindent
Abstract of this course
\end{abstract}

\section{Classical Fields}


\section{Symmetries and conservation laws}
    \subsection{Noether's theorem}
    \theorem{Noether's theorem}{there is a \textbf{conserved current} associated with every continuous symmetry of the Lagrangian}
    \subsection{Symmetries and conserved currents}
    \subsection{Global phase symmetry}
    Consider the Klein-Gordon Lagrangian density for a complex field:
    \begin{align*}
        L_{KG}=\partial_\mu\psi^*\partial^\mu\psi-m^2\psi^*\psi
    \end{align*}\\
    We can then find the conserved Noether current, as well as the conserved charge 

    \subsection{Local phase (gauge) symmetry}
    We now allow the phase change $\epsilon$ to be dependent on the space-time coordinates $x^\mu$. 
    We realise that electromagnetic fields/ covariant derivative is an essential requirement for a complex filed to remain invariant under local phase transformation.
    \subsection{Electromagnetic interaction}
    Expanding the Klein-Gordon equation:
    \begin{align*}
        L_{KG}&=(D_\mu\psi)^*(D^\mu\psi) - m^2\psi^*\psi\\
        &= \partial_\mu\psi^*\partial^\mu\psi-m^2\psi^*\psi+ieA_\mu[(\partial_\mu\psi)^*\psi-\psi(\partial^\mu\psi)]+e^2A_{\mu}A^\mu\psi^*\psi
    \end{align*}
    We see that the third term being the interaction term $eA\_muJ^\mu$, where $J^\mu$ is the free-field current.\\
    The 
    \subsection{Stress-energy tensor, angular momentum tensor}
    \subsection{Quantum fields}
\end{document}
