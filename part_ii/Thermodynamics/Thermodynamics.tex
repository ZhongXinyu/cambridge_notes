\documentclass[12pt,a4paper]{article}
\author{Xinyu Zhong\\Wolfson College}
\usepackage{physics, amsmath}
\usepackage{xcolor}
\usepackage[margin=0.5in]{geometry}

\title{Notes}
%\date{2nd Nov 2021}

\begin{document}

\begin{titlepage}
    \maketitle
\end{titlepage}

\tableofcontents

\newpage

\begin{abstract}
\noindent
Abstract of this course
\end{abstract}
\section{Basic of thermodynamics}

\section{Thermodynamics equilibrium}
    \subsection{Open systems}
    Open system is a system that is linked to a reservoir.\\
    Equilibrium condition is to maximize the total entropy of the system and the reservoir.\\
    Hence, introducing the ideal of Availability, which is minimized when the entropy of the universe is maximized with respect to the state of the system.
    \subsubsection{Availability}
    Availability A is defined as:
    \begin{equation*}
        dA=-T_RdS_{tot}
    \end{equation*}
    Note: $dA\leq 0$ and equilibrium is achieved at $dA=0$\\
    Given that:
    \begin{align*}
        dS_{tot} &= dS+dS_R >= 0\\
        &= dS + ...\\
        &= \dfrac{T_RdS-dU-p_RdV+\mu_RdN}{T_R}
    \end{align*}
    Some function variables sum up to U:
    \begin{equation*}
        A = U- T_RS+p_RV-\mu_R N
    \end{equation*}
    \subsubsection*{Boltzmann expression for entropy}
    \subsubsection{Availability is equivalent to useful work}

    \subsection{Close systems}
    The method we used to study closed system is to play \textbf{imaginary partition}
    to partition it into two or more systems.\\
    Equilibrium condition is to find the maximized S given a fixed value of U, or alternatively
    minimized U for a given S.
    \subsubsection{Constant temperature at equilibrium}
    \subsubsection{Constant chemical potential at equilibrium}
    \subsection{Overview of thermodynamics potentials}
    Here are some thermodynamics potential examples:\\
    Internal energy: $dU = TdS - pdV+\mu dN$\\
    Enthalpy: $dH = TdS + Vdp +\mu dN$\\
    Helmholtz Free energy: $dF = -TdS -pdV + \mu dN$\\ ...see discussion in mechanical equilibrium \\
    Gibbs Free energy: $dG = -SdT + Vdp +\mu dN$\\ ...see discussion in phase equilibrium \\
    \textbf{Grand potential : $d\phi = -SdT -pdV - Nd\mu$}\\ ...see discussion in Fermion and Boson Gas \\
    \\
    For given external conditions, the appropriate thermodynamic potential is a minimum in equilibrium:\\
    the minimization of these thermodynamic potential of the system is a direct consequence of the maximization of global energy.
    \subsubsection{Detailed discussion about each thermodynamic potential}
    ... Handout page 33 - 35
    \subsection {Phase Equilibrium}
    Consider a one component system at constant temperature, pressure and particle number, the equilibrium is that the Gibbs free energy is conserved.
    i.e. the total Gibbs free energy is minimized, $dS =0$ in a mixture of gas and liquid.
    \subsubsection {Phase Equilibrium in van de waal gas}
    \subsubsection {Clausius-Clapeyron Equation}
    \subsection {Mixture of ideal gas}
        In ideal gas, particles do not interact with each other, the thermodynamics properties are the sum of individual contribution of each species of "component".
        \dots
        \subsubsection{Chemical Equilibrium}
        \subsubsection{Equilibrium Constant}
\section{Statistical Mechanics}
    \textcolor{green}{
        Sterling Approximation: $\ln(n!)= n\ln{n} - n $\\
        $\dfrac{d\ln(n!)}{dn} = ln(n)$.
    }
\section{Classical Ideal Gas}
    In classical ideal gas, we have the probability density function and the associated partition function:
    \begin{align*}
        \rho &= \frac{e^{-\beta E({p_i,q_i})}}{Z}\\
        Z_{classical} &= \int e^{-\beta E({p_i,q_i})} \dfrac{d^3x d^3p}{(2\pi\hbar)^3}
    \end{align*}
    Note that this is 3-dimensional case.\\
    Trick to compute the integral, in example of a free particle:
    \begin{align*}
        Z_1 &= \int e^{-\beta E({p_i,r_i})} \dfrac{d^3p d^3r}{(2\pi\hbar)^3}\\
            &= \int d^3r \int e^{-\beta E({p_i,r_i})} \frac{d^p_x}{2\pi\hbar}\\
            &= V(\sqrt{\dfrac{k_BTm}{2\pi\hbar}})^3\\
            &= V/\lambda^3
    \end{align*}
    where $\lambda$ is also known as the thermal de Broglie wavelength 
    
%Lecture 6:
\section{Fermi gas}
%Lecture 10:
\section{Bose gas}
    Bose gas : quantum limit, Boson particles can take up any energy state instead of two\\
    Grand partition function:
    \begin{align*}
    \Xi_k   &= \sum_{n=0}^{\inf} e^{-\beta(\epsilon_k-\mu)^n}\\
            &=\frac{1}{1-e^{-\beta(\epsilon_k-\mu)}}
    \end{align*}
    Here $k$ specifies the energy level.\\
    The grand partition function for the whole system
    The grand potential is then
%Lecture 12
\section{Non-ideal Gas and Liquids}
\subsection{2 particle probability}
\subsection{Radial distribution function}

\subsection{Mean Energy}
\subsection{Viral}
Virial is defined as 
\begin{align*}
    \nu = -\dfrac{1}{2} \sum_i {\vb{r_i}\cdot \vb{f_i}}
\end{align*}
\subsubsection{$\braket{\nu_{ext}}$}
\subsubsection{$\braket{\nu_{int}}$}



\end{document}
