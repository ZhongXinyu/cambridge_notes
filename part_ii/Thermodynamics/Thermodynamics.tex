\documentclass[10pt,a4paper]{article}
\author{Xinyu Zhong\\Wolfson College}
\usepackage{physics, amsmath}
\usepackage{xcolor}
\usepackage[margin=0.5in]{geometry}
\usepackage{fancyhdr}
\pagestyle{fancyplain}
\fancyhf{}


%\date{2nd Nov 2021}
\lhead{\fancyplain{}{Zachary Zhong, xz447@cam.ac.uk}}
\rhead{\fancyplain{}{Thermodynamics}}
\cfoot{\fancyplain{}{\thepage{}}}
\setlength {\headheight}{15pt}

\newcommand{\definition}[3]
    {
    \textit{Definition #1: }
    \begin{center}
        {#2}
    \end{center}
    {#3}\\
    }
\newcommand{\theorem}[2]{\textbf{\textcolor{red}{#1: }}{#2}\\}
\newcommand{\example}[2]{\textbf{Example: #1}\\\textcolor{blue}{#2}\\}
\title{Notes}

\begin{document}

\begin{titlepage}
    \maketitle
\end{titlepage}

\tableofcontents

\newpage

\begin{abstract}
\noindent
Abstract of this course
\end{abstract}
\section{Basic of thermodynamics}

\section{Thermodynamics equilibrium}
    \subsection{Open systems}
    Open system is a system that is linked to a reservoir.\\
    Equilibrium condition is to maximize the total entropy of the system and the reservoir.\\
    Hence, introducing the ideal of Availability, which is minimized when the entropy of the universe is maximized with respect to the state of the system.
    \subsubsection{Availability}
    Availability A is defined as:
    \begin{equation*}
        dA=-T_RdS_{tot}
    \end{equation*}
    Note: $dA\leq 0$ and equilibrium is achieved at $dA=0$\\
    Given that:
    \begin{align*}
        dS_{tot} &= dS+dS_R >= 0\\
        &= dS + ...\\
        &= \dfrac{T_RdS-dU-p_RdV+\mu_RdN}{T_R}
    \end{align*}
    Some function variables sum up to U:
    \begin{equation*}
        A = U- T_RS+p_RV-\mu_R N
    \end{equation*}
    \subsubsection*{Boltzmann expression for entropy}
    \subsubsection{Availability is equivalent to useful work}

    \subsection{Close systems}
    The method we used to study closed system is to play \textbf{imaginary partition}
    to partition it into two or more systems.\\
    Equilibrium condition is to find the maximized S given a fixed value of U, or alternatively
    minimized U for a given S.
    \subsubsection{Constant temperature at equilibrium}
    \subsubsection{Constant chemical potential at equilibrium}
    \subsection{Overview of thermodynamics potentials}
    Here are some thermodynamics potential examples:\\
    Internal energy: $dU = TdS - pdV+\mu dN$\\
    Enthalpy: $dH = TdS + Vdp +\mu dN$\\
    Helmholtz Free energy: $dF = -TdS -pdV + \mu dN$\\ ...see discussion in mechanical equilibrium \\
    Gibbs Free energy: $dG = -SdT + Vdp +\mu dN$\\ ...see discussion in phase equilibrium \\
    \textbf{Grand potential : $d\phi = -SdT -pdV - Nd\mu$}\\ ...see discussion in Fermion and Boson Gas \\
    \\
    For given external conditions, the appropriate thermodynamic potential is a minimum in equilibrium:\\
    the minimization of these thermodynamic potential of the system is a direct consequence of the maximization of global energy.
    \subsubsection{Detailed discussion about each thermodynamic potential}
    ... Handout page 33 - 35
    \subsection {Phase Equilibrium}
    Consider a one component system at constant temperature, pressure and particle number, the equilibrium is that the Gibbs free energy is conserved.
    i.e. the total Gibbs free energy is minimized, $dS =0$ in a mixture of gas and liquid.
    \subsubsection {Phase Equilibrium in Van de Waal gas}
    \subsubsection {Clausius-Clapeyron Equation}
    \subsection {Mixture of ideal gas}
        In ideal gas, particles do not interact with each other, the thermodynamics properties are the sum of individual contribution of each species of "component".
        \dots
        \subsubsection{Chemical Equilibrium}
        \subsubsection{Equilibrium Constant}
\section{Statistical Mechanics}
    \begin{theorem}
        {Sterling Approximation:}
        {
            \begin{align*}
                \ln(n!)= n\ln{n} - n \\
                \dfrac{d\ln(n!)}{dn} = ln(n).
            \end{align*}
            }
    \end{theorem}
        
\section{Classical Ideal Gas}
    In classical ideal gas, we have the probability density function and the associated partition function:
    \begin{align*}
        \rho &= \frac{e^{-\beta E({p_i,q_i})}}{Z}\\
        Z_{classical} &= \int e^{-\beta E({p_i,q_i})} \dfrac{d^3x d^3p}{(2\pi\hbar)^3}
    \end{align*}
    Note that this is 3-dimensional case.\\
    Trick to compute the integral, in example of a free particle:
    \begin{align*}
        Z_1 &= \int e^{-\beta E({p_i,r_i})} \dfrac{d^3p d^3r}{(2\pi\hbar)^3}\\
            &= \int d^3r \int e^{-\beta E({p_i,r_i})} \frac{d^p_x}{2\pi\hbar}\\
            &= V(\sqrt{\dfrac{k_BTm}{2\pi\hbar}})^3\\
            &= V/\lambda^3
    \end{align*}
    where $\lambda$ is also known as the thermal de Broglie wavelength 
    
%Lecture 6:

%lecture 7:
\subsection{Chemical equilibrium}
\subsection{High-T limit}
\subsection{Chemical potential}

\subsection{Cross-over to quantum limit}
%lecture 8:
Continues on quantum ideal gas
\section{Fermi gas}
%lecture 9:
\subsection{Fermi gas in low/medium/high Temperature}
%Lecture 10:
\section{Bose gas}
    Bose gas : quantum limit, Boson particles can take up any energy state instead of two\\
    Grand partition function:
    \begin{align*}
    \Xi_k   &= \sum_{n=0}^{\inf} e^{-\beta(\epsilon_k-\mu)^n}\\
            &=\frac{1}{1-e^{-\beta(\epsilon_k-\mu)}}
    \end{align*}
    Here $k$ specifies the energy level.\\
    The grand partition function for the whole system
    The grand potential is then
%lecture 11
\subsection{Quasi-particle excitation}
\subsection{Photons}
\subsection{Spin Waves}
\subsection{Phonon and Debye model}
%Lecture 12
\section{Non-ideal Gas and Liquids}
\subsection{2 particle probability}
\subsection{Radial distribution function}

\subsection{Mean Energy}
\subsection{Virial}
Virial is defined as: 
\begin{align*}
    \nu = -\dfrac{1}{2} \sum_i {\vb{r_i}\cdot \vb{f_i}}
\end{align*}
\theorem{Virial Theorem}{Virial Theorem states that the mean virial is equal to the mean kinetic energy.}
\subsubsection{${\nu_{ext}}$}
\subsubsection{${\nu_{int}}$}
\subsection{Virial Expansion}
Virial Expansion

\section{Phase Equilibrium and Transition}
%lecture 13
\subsection{Ising model}

%lecture 14
\subsection{Landau Theory}
\subsubsection{Critical behavior}
\subsubsection{External field}
\subsection{2nd order transition}
%lecture 15
\subsection{1st order transition}
\subsubsection{Calculation of $T*$,$T_1$ and latent heat}
%lecture 16
\section{Fluctuation}
\begin{align*}
    \langle \Delta x^2\rangle = \langle x^2\rangle -\langle x\rangle ^2 
    =\dfrac{1}{Z}\sum_i x_i^2 e^{-E_i/k_BT} - (\frac{1}{Z}\sum_i x_i e^{-E_i/k_BT})^2
\end{align*}
If there is a simple dependence of energy on the variable, 
e.g. if such dependence is linear, $E_i=-fx_i$, then a general trick applies:
\begin{align*}
    \langle x\rangle &= \dfrac{1}{Z}\sum_i x_i e^{fx_i/k_BT} = \frac{1}{\beta Z}(\frac{\partial Z}{\partial f})\\
    \langle x\rangle &= \dfrac{1}{Z}\sum_i x_i^2 e^{fx_i/k_BT} = \frac{1}{\beta^2 Z}(\frac{\partial^2 Z}{\partial f^2})
\end{align*}
We realize that:
$\langle \Delta x^2\rangle = \langle x^2\rangle -\langle x\rangle ^2 = ... = k_BT(\frac{\partial <x>}{\partial f})$\\
\definition{response function}{
    $\dfrac{\partial \langle \Delta x^2\rangle}{\partial y}$
}{as the response function.}
\begin{example}
    {Spring with a force f and displacment x}
    {... see notes.}
\end{example}
\begin{example}
    {The fluctuation in the magnetisation of a subsystem consisting of a paramagnet in an external filed B, in and contact with a reservoir at time T.}
    {... see handout}
\end{example}
\subsection{Thermodynamics}
In a large system, the curvature of the availability function is related to the size of the fluctuations.\\
Probability density function can be written as an exponential function of A.\\
We can then 
The probability distribution is actually a Gaussian approximation w.r.t
\section{Stochastic physics}
    \begin{theorem}
        {White noise properties}{
        $\expval{\xi(t)}=0$ but $\expval{\xi(t)^2}=0$ and  $\expval{\xi(t_1)\xi(t_1)}=0$ if $t_1\neq t_2$.
        \begin{equation*}
            \expval{\xi(t)\xi(t')}=\Gamma\delta(t-t')
        \end{equation*}
        }
    \end{theorem}
    \\
    \begin{theorem}
        {Fluctuation dissipation theorem}{
        ???
        }
    \end{theorem}
    \subsection{Damped Oscillator}
        \subsubsection{Langevin equation}
            \begin{theorem}{Langevin equation}
                {
                Fluctuating system obeys the damped single harmonic oscillator equation with random force $\epsilon(t)$
                \begin{align*}
                    m\ddot{x}+\lambda\dot{x}+kx=\epsilon(t)
                \end{align*}
                }
            \end{theorem}
    \subsection{Diffusion}
    \subsection{Diffusion in external potentials}
    \begin{example}
        {Derivation of kinetic equation of $P(x,t)$, with external potential $V(x)$}
        {
        {Random walk from bias}\\
        {From analogy with sediment problem in gravity, $V=mgh$} 
        }
    \end{example}
    \subsection{Kramers problem: escape over a potential barrier}
\end{document}
