\documentclass[12pt,a4paper]{article}
\author{Xinyu Zhong\\Wolfson College}
\usepackage{physics, amsmath}
\usepackage{xcolor}
\usepackage[margin=0.5in]{geometry}

\title{Notes}
%\date{2nd Nov 2021}

\usepackage{fancyhdr}
\pagestyle{fancyplain}
\fancyhf{}
\lhead{\fancyplain{}{Zachary Zhong, xz447@cam.ac.uk}}
\rhead{\fancyplain{}{Thermodynamics}}
\cfoot{\fancyplain{}{\thepage{}}}
\setlength {\headheight}{15pt}

\newcommand{\definition}[3]
    {
    \textit{Definition #1: }
    \begin{center}
        {#2}
    \end{center}
    {#3}\\
    }
\newcommand{\theorem}[2]{\textbf{\textcolor{red}{#1: }}\textcolor{red}{#2}}
\newcommand{\example}[1]{\par\textbf{Example: }\textcolor{blue}{#1}}

\begin{document}

\begin{titlepage}
    \maketitle
\end{titlepage}

\tableofcontents

\newpage

\begin{abstract}
\noindent
Abstract of this course
\end{abstract}
\section{Introduction}
mention supervised vs. unsupervised, supervised is more relavent for physics.
\section{Overview of fitting}
Average\\
Straight line\\
Tayler expansion only works for small x, disaster at large x\\
Pade approximant $\frac{ax^2+bx^3+...}{c+dx+...+x^6}$ making sure that f(x) does not tend to infinity at large x, in this case tend to 1/x\\
After 150 years or so \\
Neuro network, $\frac{x}{1+a\abs{x}}$ using less indicator\\
after 30 years\\
Deep neuro network, layers of sum of indicator
the deepness refers to the layers

\section{Different methods in Machine learning}
Random forest/Gaussian etc.
\section{Real life example}


\end{document}